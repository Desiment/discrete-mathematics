\documentclass{article}
\usepackage{iftex}

%% Разрешить компиляцию только с движком LuaTex
\ifLuaTeX
\else
    \newlinechar 64\relax
    \errorcontextlines -1\relax
    \immediate\write20{@
        ************************************************@
        * LuaLaTex is required to compile this document.@
        * Sorry!@
        ************************************************}%
    \batchmode\read -1 to \@tempa
\fi

%% Для русификации достаточно подключить пакет fontspec и
%% выбрать Unicode шрифт в котором есть кириллические глифы. Ниже
%% основным шрифтом выбирается Unicode версия шрифта Computer Modern с заcечками
\usepackage{fontspec}
\setmainfont{CMU Serif}
\setsansfont{CMU Sans Serif}
\setmonofont{CMU Typewriter Text}

%% В XeLaTex или LuaLaTeX альтернативой известного пакета babel является пакет polyglossia.
%% Теперь у нас будут переносы слов
\usepackage{polyglossia}
\setdefaultlanguage{russian}
\setotherlanguage{english}

\usepackage[autostyle]{csquotes} % Правильные кавычки в зависимости от языка
\usepackage{totcount}
\usepackage{setspace}

% ToDo:
% [] MathNote   : \sum[] syntax
% [] MathNote   : \tikzmatrix 
% [] Layout     : A4 geometry
% [] Layout     : Define colors
% [] References : Load knowledge
% [] References : Create simple clever refs
% [] Fix hidding enviroment

%Отключить предупреждения об кастномной использовании пакетов "You have requested package..."
\usepackage{silence}
\WarningFilter{latex}{You have requested package}

\usepackage{xparse}
\usepackage{configuration/floats}
\usepackage{configuration/layout}
\usepackage{configuration/references}
\usepackage{configuration/mathnote}

\sloppy
% Окружения для набора задач
\newcounter{boxlblcounter}  

\newcommand{\task}[1]{\fbox{\begin{minipage}{4em}\centering\it #1\end{minipage}}}
\newcommand{\makeboxlabel}[1]{\fbox{\begin{minipage}{2em}\centering\it #1\end{minipage}}\hfill}% \hfill fills the label box
\newenvironment{tasklist}
  {\begin{list}
    {\arabic{boxlblcounter}}
    {\usecounter{boxlblcounter}
     \setlength{\labelwidth}{3em}
     \setlength{\labelsep}{0em}
     \setlength{\itemsep}{2pt}
     \setlength{\leftmargin}{1.5cm}
     \setlength{\rightmargin}{2cm}
     \setlength{\itemindent}{0em} 
     \let\makelabel=\makeboxlabel
    }
  }{\end{list}}


\newboolean{ShowHint}
\newboolean{ShowSolution}

% Подсказка
\newcommand{\hint}[1]{\ifthenelse{\boolean{ShowHint}}{\noindent\rotatebox[origin=c]{180}{\noindent
\begin{minipage}[t]{\linewidth} \noindent \it Указание: #1 \end{minipage}}}{}}

% solution: окружение для набора решений
% solution*: форсировано показывает решение, независимо от флага
\makeatletter
\newenvironment{solution*}[1][\text{Решение}]{ 
  \par
  \pushQED{\qed}%
  \normalfont
  \topsep0pt \partopsep3pt
  \trivlist
  \item[\hskip\labelsep
    \itshape
    #1\@addpunct{.}]\ignorespaces
}{
  \ifthenelse{\boolean{ShowSolution}}{
    \popQED\endtrivlist\@endpefalse
    \addvspace{6pt plus 6pt} % some space after
  }{\end{hidden}}
}
\makeatother

%https://tex.stackexchange.com/questions/533218/hiding-an-environment-that-contains-minted-code
%https://tex.stackexchange.com/questions/38150/in-lualatex-how-do-i-pass-the-content-of-an-environment-to-lua-verbatim
\RequirePackage{luacode}
\begin{luacode*}
do 
    function eat_buffer(buf)
        i,j = string.find(buf,"\\end{solution}")
        if i==nil then return "" else return nil end
    end
    function start_proccesing_solution(eat)
        if eat then luatexbase.add_to_callback('process_input_buffer', eat_buffer, 'eat_buffer') end
    end
    function stop_proccesing_solution(eat)
        if eat then luatexbase.remove_from_callback('process_input_buffer', 'eat_buffer') end
    end
end
\end{luacode*}
%https://tex.stackexchange.com/questions/537219/conditionals-inside-newcommand-with-empty-argument
%https://tex.stackexchange.com/questions/63223/xparse-empty-arguments
\ExplSyntaxOn
\DeclareExpandableDocumentCommand{\IfNoValueOrEmptyTF}{mmm}{\IfNoValueTF{#1}{#2}{\tl_if_empty:nTF {#1} {#2} {#3}}}
\NewDocumentCommand{\DefaultName}{mm}{\IfNoValueOrEmptyTF{#1}{#2}{#1}}
\ExplSyntaxOff

% WARNING!
% Buggy: one have always have
% \begin{solution}{NAME}
%   ....
% \end{solution}
% EVEN if NAME is empty
% if one starts some text on the same line as \begin{solution} or \end{solution} it could'not be hidden
\newenvironment{solution}[1]%
{
  \ifthenelse{\boolean{ShowSolution}}
    {\begin{solution*}[\DefaultName{#1}{\textit{Решение}}]\directlua{flag_eat=false}}
    {\directlua{flag_eat=true}}
  \directlua{start_proccesing_solution(flag_eat)}
}
{
  \ifthenelse{\boolean{ShowSolution}}{\end{solution*}}{}
  \directlua{stop_proccesing_solution(flag_eat)}
}


\mmzset{memo dir = images/cache/\jobname}

\title{Отношения и отображения}
\author{Михаил Михайлов}
\date{\today}

\begin{document}
\maketitle
\tableofcontents
\newpage

\section{Отношения, отображения и соответствия}

\subsection{Отношения}
\begin{definition}
    \label{def:relation}
    Пусть \(X_1, X_2, \ldots, X_n\) --- фиксированные множества.
    
    \begin{enumerate}
        \item \textbf{Отношением \(R\) на \(X_1, X_2, \ldots, X_n\)} называется подмножество \(X_1 \times \ldots \times X_n\). Факт связи \(x_1, \ldots, x_n\) обозначается как \(R(x_1, \ldots, x_n)\) или \(x_1 R x_2\), если \(n = 2\).
        \item \(R\) --- \(n\)-арное отношение на \(X\), если отношение \(R\) задано на \(X^n\).
        \item \textbf{Диагональю} называется подмножество \(X^n\): \(\Delta \coloneq \set{(x,\ldots, x)\,|\, x \in X}\). Диагональ на \(X^2\) --- отношение равенства.
        \item Бинарное отношение на \(X\) называется:
            \begin{itemize}
                \item \textbf{рефлексивным}, если: \(\forall x \in X: x R x \);
                \item \textbf{симметричным}, если: \(\forall x, y \in X: (x R y \Rightarrow y R x)\);
                \item \textbf{антисимметричным} если: \(\forall x, y \in X: x R y \,\land\, y R x \Rightarrow x = y\);
                \item \textbf{транзитивным}, если: \(\forall x, y, z \in X: (x R y \land y R z \Rightarrow x R z)\).
            \end{itemize}
    \end{enumerate}
\end{definition}
    
\begin{definition}
    \label{def:relation-fields}
    Пусть \(R\) --- отношение на \(X, Y\)
    \begin{enumerate}
        \item \textbf{Областью определения} \(R\) называется множество \(\dom{R}\) всех \(x \in X\), таких что для них есть хотя бы один \(y \in Y\) такой что \(x R y\) 
        \[\dom{R} \coloneq \set{x \in X\,|\, \exists y \in Y: (x, y) \in R }\]
        \item \textbf{Областью значений} \(R\) называется множество \(\ran{R}\) всех \(y \in Y\), таких что для них есть хотя бы один \(x \in X\) такой что \(x R y\)  
        \[\ran{R} \coloneq \set{y \in Y\,|\, \exists x \in X: (x, y) \in R }\]
        \item \textbf{Областью действия} \(R\) называется множество \(\rfield{R} \coloneq \dom{R} \cup \ran{R}\)
    \end{enumerate}
\end{definition}
    
\subsubsection{Примеры отношений}
    TBD 


\subsection{Отображения и соответствия}
\begin{definition}  
    \label{def:maps}
    Пусть \(X, Y\) --- множества. \(R\) --- отношение на \(X \times Y\)

    Отношение \(R\) называют \textbf{соответствием из множества \(X\) в множество \(Y\)}
    
    \textbf{Отображение} --- соответствие \(D\) такое, что "для каждого \(x\) есть только один \(y\)"\,: \[\forall\, a \in X: \exists!\, b \in Y: (a, b) \in D\].
\end{definition}
\begin{defnotation}
    Отображение \(f\) из \(X\) в \(Y\) мы будем обозначать как \(\MapTo{f}{X}{Y}\). 
    
    Если известно какому элементу какой сопоставляется можно использовать нотацию вида:
    \begin{align*}
        &\MapTo{f}{X}{Y}\\
        &x \mapsto y
    \end{align*}
\end{defnotation}

\begin{definition}
    \label{def:image-preimage}
    Пусть \(f: X \rightarrow Y\) отображение:
    \begin{itemize}
        \item \textbf{Образ \(A \subset X\)} --- множество всех \(y \in Y\) "получаемых из \(A\)"\,: \(f(A) = \set{y \in Y \,|\, y = f(x), x \in A}\)
        \item \textbf{Прообраз \(B \subset Y\)} --- множество всех \(x \in X\) "для которых образ есть \(y\) из \(B\)"\,: \(f^{-1}(B) = \set{x \in X | f(x) \in B}\)
    \end{itemize}
\end{definition}

\begin{definition}
    \label{def:inj-surj-bij-maps}
        Пусть \(f: X \rightarrow Y\) отображение:
    \begin{itemize}
        \item \(f\) \textbf{инъекция} если: 
        \[\forall\, a_1, a_2 \in X: f(a_1) = f(a_2) \Rightarrow a_1 = a_2\]
        \item \(f\) \textbf{сюръекция} если
        \[\forall\, b \in Y: f^{-1}(b) \neq \varnothing\]
        \item \(f\) \textbf{биекция} если \(f\) оно сюръекция и инъекция одновременно.
    \end{itemize}
\end{definition}

\begin{proposition}[Свойства образа]
    \label{prop:image-properties}
    Пусть \(f: X \rightarrow Y\). Тогда 
    \begin{itemize}
        \item \(\forall A, B \subset X\): \(f(A \cup B) = f(A) \cup f(B)\)
        \item \(\forall \set{A_i}_{i \in I},\, A_i \subset X\):  \(f(\bigcup_{i \in I} A_i) = \bigcup_{i \in I} f(A_i)\)
    \end{itemize}
    Так же, \(f\) инъективно тогда и только тогда, когда:  \(\forall A, B \subset X\): \(f(A \cap B) = f(A) \cap f(B)\)
\end{proposition}
\begin{proof}
    Тривиально (используется прямое вычисление и определения)
\end{proof}

\begin{proposition}[Свойства прообраза]
    \label{prop:preimage-properties}
    Пусть \(f: X \rightarrow Y\). Тогда 
    \begin{itemize}
        \item \(\forall A, B \subset Y\): \(f^{-1}(A \cup B) = f^{-1}(A) \cup f^{-1}(B)\)
        \item \(\forall A, B \subset Y\): \(f^{-1}(A \cap B) = f^{-1}(A) \cap f^{-1}(B)\)
        \item \(\forall A, B \subset Y\): \(f^{-1}(A \setminus B) = f^{-1}(A) \setminus f^{-1}(B)\)
        \item \(\forall A, B \subset Y\): \(f^{-1}(A \Delta B) = f^{-1}(A) \Delta f^{-1}(B)\)
        \item \(\forall \set{A_i}_{i \in I},\, A_i \subset Y\):  \(f^{-1}(\bigcup_{i \in I} A_i) = \bigcup_{i \in I} f^{-1}(A_i)\)
        \item \(\forall \set{A_i}_{i \in  I},\, A_i \subset Y\):  \(f(\bigcap_{i \in I} A_i) = \bigcap_{i \in  I} f(A_i)\)
    \end{itemize}
\end{proposition}
\begin{proof}
    Тривиально (используется прямое вычисление и определения).
\end{proof}

\begin{definition}
    \label{def:comp-map}
    Пусть дана последовательность отображений: \(X \xrightarrow[]{f} Y \xrightarrow[]{g} Z\). Тогда можно определить композицию \(g \circ f: X \rightarrow Z,\, x \mapsto g(f(x))\). 
\end{definition}

Определение на языке теории множеств
\begin{definition}
    \label{def:comp-rel}
    Пусть \(\Delta \subset X \times Y\), \(\Gamma \subset Y \times Z\) --- соответствия. Их композиция: \[\Delta \circ \Gamma = \set{(x, z) \in X \times Z | \exists\, y: (x, y) \in \Delta \land (y, z) \in \Gamma}\]
\end{definition}
\begin{remark}
    \label{rem:notation}
    Нотация записи слева-направо (\(\Delta \circ \Gamma\)) более естественна так как в большой последовательности функций удобнее думать про то, какое отображение применяется первым, а не последним. Однако, в нотации в смысле отображений, мы будем пользоваться записью справа-налево
\end{remark}

\begin{proposition}
    \label{prop:comp-ass}
    Композиция отображений ассоциативная операция. Пусть заданы отображения \(X \xrightarrow[]{f} Y \xrightarrow[]{g} Z \xrightarrow[]{h} T\). Тогда \(h \circ (g \circ f) = (h \circ g) \circ f\).
\end{proposition}
\begin{proof}
    Пусть \(y = g(h(x))\):
    \[h \circ (g \circ f)  = f(g(h(x))) = f(y)\]
    \[(h \circ g) \circ f = f(y)\]
\end{proof}

\begin{?}
    Докажите предложение \ref{prop:comp-ass} в смысле определения \ref{def:comp-rel}.
\end{?}

\begin{definition}
    \label{def:inverse-map}
    \textbf{Тождественным отображением} \(\mathrm{id}_X: X \rightarrow X\) отображение такое, что \(\forall x \in X:\, \mathrm{id}_X(x) = x\). Эквивалентное определение \(\mathrm{id}_X = \set{(x, x) | x \in X} \subset X \times X\).    \\
    Пусть \(f: X \rightarrow Y,\ g: Y \rightarrow X\).
    \begin{itemize}
        \item \(g\) --- \textbf{левое обратное} отображение к \(f\) если \(g \circ f = \mathrm{id}_X\)
        \item \(g\) --- \textbf{правое обратное} отображение к \(f\) если \(f \circ g = \mathrm{id}_Y\)
        \item \textbf{Обратное отображение} \(g: g \circ f = \mathrm{id}_X \land f \circ g = \mathrm{id}_Y\).
    \end{itemize}
\end{definition}

\begin{theorem}[о существовании обратных отображений]
    \label{th:inverse-map-existance}
    Пусть \(X \neq \varnothing\)
    У отображения \(f: X \rightarrow Y\) есть левое обратное тогда и только тогда, когда \(f\) --- инъекция. У отображения \(f\) есть правое обратное отображение тогда и только тогда, когда \(g\) --- сюръекция. \(f\) обратимо если \(f\) --- биекция. 
\end{theorem}
\begin{proof} Докажем в каждую из сторон:
    \begin{itemize}
        \item Если \(f\) --- обратима слева, то \(f\) инъективна. Имеем \(\exists\, g: g \circ f = \mathrm{id}_X\). Если \(f\) --- инъекция, то \(f(x_1) = f(x_2) \Rightarrow x_1 = x_2\). Но \[f(x_1) = f(x_2) \Rightarrow g(f(x_1)) = g(f(x_2)) \Rightarrow \mathrm{id}_X x_1 = \mathrm{id}_X x_2 \Rightarrow x_1 = x_2\]
        \item Если  \(f\) инъективна, то \(f\) обратима слева. \(Y_1 = Y \setminus f(X)\). Выберем \(x_0 \in X\) Определим
        \[
        g(y) = \begin{cases} x: \set{x} = f^{-1}(y), y \in f(X) \ \text{т.к. в этом случае прообраз будет состоять из одного элемента} \\
        x_0, y \in Y_1 \end{cases}
        \]
        Тогда \(g \circ f = \mathrm{id}_X\)
        \item Если \(f\) --- обратима справа, то \(f\) сюръективна. Имеем \(\exists\, g: f \circ g = \mathrm{id}_Y\). Пусть \(y_0 \in Y\). Проверим, что \(g(y_0) \in f^{-1}(y_0)\): 
        \[g(y_0) \in f^{-1}(y_0) \Leftrightarrow f(g(y_0)) \in \set{y_0} \Leftrightarrow f(g(y_0)) = y_0\]
        \item Если \(f\) --- сюръективна, то \(f\) обратима справа. Возьмем \(y_0 \in Y\). Определим \(g(y_0)\) --- возьмем из множества \(f^{-1}(y_0)\) один элемент \(x_0\). Полагаем \(g(y_0) = x_0\)
        \[(f \circ g)(y_0) = f(g(y_0)) = f(x_0) = y_0 = \mathrm{id}_Y(y_0)\]
        То есть \(f \circ g = \mathrm{id}_Y\)
    \end{itemize}
\end{proof}

\begin{proposition}
    \label{prop:comp-properties}
    Свойства композиции. Пусть \(f: X \rightarrow Y, g: Y \rightarrow Z\).
    \begin{enumerate}
        \item Если \(f\), \(g\) --- инъективны, то \(g \circ f\) --- иънективно.
        \item Если \(f\), \(g\) --- cюръективны, то \(g \circ f\) --- сюръективно.
        \item Если \(g \circ f\) --- сюръективна, то \(g\) --- сюръективна.
        \item Если \(g \circ f\) --- инъективно, то \(f\) --- инъективна.
        \item \((g \circ f) (A) = g(f(A)), \forall\, A \subset X\).
        \item \((g \circ f)^{-1}(A) = f^{-1}(g^{-1}(A)), \forall\, A \subset Z\)
        \item Если \(f, g\) обратимы, то \(g \circ f\) обратима. 
    \end{enumerate}
\end{proposition}
\begin{proof} Докажем каждое свойство: 

    \textbf{Свойство 1}: По определению. 
    
    \textbf{Свойство 2}: По определению. 
    
    \textbf{Свойство 3}: Если \(g\) не сюръективна, то существует \(z \in Z\) такой, что \(g^{-1}(z) = \varnothing\). Но так как \(g \circ f\) сюръективна, для \(z\) существует по крайне мере один \(x\) такой что \(g(f(x)) = z\). Тогда \(f(x) \in g^{-1}(z)\). Противоречие.
    
    \textbf{Свойство 4}: Если \(f\) не инъективна, то \(\exists x_1 \ne x_2: f(x_1) = f(x_2)\), но тогда \(g \circ f (x_1) = g \circ f (x_2)\), т.е. \(g \circ f\) не инъективно. Противоречие. 
    
    \textbf{Свойство 5}: По определению. 
    
    \textbf{Свойство 6}: В силу свойства 5 имеем \((g \circ f)(A) = g(f(A)) \Rightarrow (g \circ f)^{-1}(Z) = f^{-1} \circ g^{-1} = f^{-1}\left(g^{-1}(Z)\right)\)
    
    \textbf{Свойство 7}: Известно, что функция обратима тогда и только тогда, когда она инъективна (см. теорему \ref{th:inverse-map-existance}), а значит свойство равносильно свойству 1.
\end{proof}


\begin{remark*}
    Можно говорить об соответствиях и отображения, область действия которых является собственным классом, т.е. $\rfield{f}$ не является множеством. Для того чтобы это делать, необходимо ввести понятие упорядоченной пары двух собственных классов (что будет сделано ниже) и сказать что TBD 
\end{remark*}

\subsection{Отношения эквивалентности}
    
    \begin{definition}
        \label{def:equiv-rel}
        \(X\) --- фиксированное множество, \(R\) --- бинарное отношение на \(X\)
        \begin{itemize}
            \item Отношение \(R\) на \(X\) называется \textbf{отношением эквивалентности}, если оно симметрично, транзитивно, рефлективно.
             \item Пусть \(R\) --- отношение эквивалентности на \(X\). \textbf{Классом эквивалентности} \({[x_0]}_R\) элемента \(x_0 \in X\) называется множество \(\set{x \in X | x R x_0}\). 
        \end{itemize}
    \end{definition}
    
    \begin{definition}
        \label{def:equiv-rel-map}
        Пусть задано отображение \(f: X \rightarrow Y\). Отношение \(x_1 \sim x_2 \xLeftrightarrow[]{\text{\tiny def}} f(x_1) = f(x_2)\) мы будем называть отношением \textbf{равнообразности}.
    \end{definition}
    \begin{remark}
        \label{rem:equiv-rel-map}
        Равнообразность --- отношение эквивалентности. 
    \end{remark}
    
    \begin{proposition}
        \label{prop:equiv-classes-intersect}
        Пусть \(\sim\) --- отношение эквивалентности на множестве \(X\). Тогда 
        \begin{enumerate}
            \item Если \(a \sim b,\ a, b \in X\), то \([a] = [b]\).
            \item Если \(a \not \sim b\), то \([a] \cap [b] = \varnothing\). 
        \end{enumerate}
    \end{proposition}
    \begin{proof}
        Пусть \(x \in X\) и:
        \begin{itemize}
            \item \(a \sim b\) и \(x: x \in [a] \land x \not\in [b]\). Так как \(x \sim a\) и \(a \sim b\), то \(x \sim b\). Но тогда \(x \in [b]\), противоречие условию. Значит таких \(x\) нет. 
            \item \(a \not\sim b\) и \(x \in [a] \cap [b]\). Но тогда \(a \sim x\) и \(b \sim x\), а значит \(a \sim b\), противоречие условию. Значит таких \(x\) нет. 
        \end{itemize}
    \end{proof}
    
    \begin{corollary}
        \label{rem:quotient-set-partition}
        \(X\) является объединением не пересекающихся классов эквивалентности по отношению \(\sim\). Иначе говоря, \(\sim\) задает разбиение \(X\) на классы эквивалентности. 
    \end{corollary}
    
    \begin{definition}
        \label{def:transerval}
        \textbf{Трансерваль} ---- множество, содержащие по одному представителю из каждого класса эквивалентности. 
    \end{definition}
    
    \begin{definition}
        \label{def:quotient-set}
        Пусть \((X, \sim)\) --- множество с отношением эквивалентности. Множество, элементы которого --- классы эквивалентности по отношению \(\sim\) называется \textbf{фактормножеством \(X\) по отношению \(\sim\)} и обозначается \(X / \sim\). 
    \end{definition}
    
    \begin{definition}
        \label{def:qutioent-set-canonical-map}
        \textbf{Каноническая проекция на фактормножество} --- отображение \\ \(\pi: X \surto X / \sim,\  x \mapsto [x]\)
    \end{definition}
    
    Если \(R\) --- рефлексивное транзитивное отношение, то отношение \(x \sim_R y \Leftrightarrow xRy \land yRx\) будет отношением эквивалентности 
    
    \begin{definition}
        Пусть \((X, \sim)\) --- множество с отношением эквивалентности, \(Y\) --- некоторое множество. Отображение \(f \colon X \to Y\) называется \(\sim\)~\textbf{инвариантным} если оно принимает одинаковые отношения на классах эквивалентности, иначе говоря если 
        \[
        x \sim y \Rightarrow f(x) = f(y)
        \]
    \end{definition}
    \begin{proposition}[Универсальное свойство фактормножества)]
    Пусть \((X, \sim)\) --- множество с отношением эквивалентности. Тогда для любого множества \(Y\) и \(\sim\)~\textbf{инвариантного} отображения  \(f \colon X \to Y\) существует единственное отображение $\overline{f}$ такое что следующая диаграмма коммутативна:
        \begin{center}
            \tikzcdset{row sep/normal=1cm}
            %Todo
            \begin{tikzcd}
                X \arrow{rd}{f} \arrow[twoheadrightarrow]{r}{\pi}  & X/\sim \arrow{d}{\overline{f}} \\  & Y
            \end{tikzcd}
        \end{center}
        То есть что \(f = \pi \circ \overline{f}\).
    \end{proposition}
    \begin{proof}
        Тривиально, выбором трансервали (что возможно благодаря \ref{ZFC-Choice}).
    \end{proof}
    
\subsection{Отношения порядка}
    \begin{definition}
        \label{def:order-relations}
        Отношение \(R\) называется отношением \textbf{предпорядка} если оно рефлексивно и транзитивно.
        \begin{itemize}
            \item Предпорядок \(R\) называется \textbf{частичным порядком}, если \(R\) антисимметрично.
            \item Частичный порядок \(R\) называется \textbf{линейным порядком}, если любые два элемента сравнимы.
        \end{itemize}
    \end{definition}
    
    \begin{remark}
        \label{rem:parent-set-of-relation}
        Важно помнить на каком множестве определено отношение порядка. Например, на множестве \(\mathbb N\) отношение делимости антисимметрично, а на множестве \(\mathbb Z\) --- нет.
    \end{remark}
    
    \begin{proposition}
        \label{prop:antisym-sym-relation}
        Если \(R\) - антисимметричное и симметричное одновременно, то \(R \subset \Delta\)
    \end{proposition}
    \begin{proof}
        Очевидно, что \(aRb \Rightarrow a = b\). 
    \end{proof}
    
    \begin{definition}
        \label{def:maximal-minimal-elements}
        Пусть \((X, \prec)\) --- частично упорядоченное множество.  \(A \subset X\).
        \begin{itemize}
            \item Элемент \(m\) называется \textbf{наименьшим} элементом \(A\) если \(\forall\, x \in A: m \prec x\)
            \item Элемент \(m\) называется \textbf{минимальным} элементом \(A\) если \(\forall\, x \in A: x \prec m \Rightarrow x = m\)
            \item Элемент \(m\) называется \textbf{наибольшим} элементом \(A\) если \(\forall\, x \in A: m \succ x\)
            \item Элемент \(m\) называется \textbf{максимальным} элементом \(A\) если \(\forall\, x \in A: x \succ m \Rightarrow x = m\)
        \end{itemize}
    \end{definition}
    
    \subsubsection{Отношения предпорядка}
    \begin{definition}
        Пусть \((X, R)\) --- произвольное отношение на множестве \(X\). Наименьшее по включение отношение \(S\) такое что \(R \subseteq S\) и \(S\) --- отношение предпорядка называется \textbf{рефлексивно-транзитивное замыканием} отношения \((X, R)\). 
    \end{definition}
    \begin{remark}
    При этом, замыкание не обязательно будет антисимметрично.
    \end{remark}
        \begin{definition}
        \textbf{Диаграмма Хассе} отношения \(R\) --- наименьшее отношение \(S\), такое что рефлексивно-транзитивное замыкание \(S\) это \(R\).
    \end{definition}
    
    \begin{figure}[h!]
        \centering
        \begin{tikzpicture}
        [scale=.3,auto=left,every node/.style={circle,fill=blue!20}]
        \node (n1) at (4,10) {a};
        \node (n2) at (1,7)  {b};
        \node (n3) at (7,7)  {c};
        \node (n4) at (1,4) {d};
        \node (n5) at (7,4)  {e};
        \node (n6) at (4,1)  {f};
        \foreach \from/\to in {n1/n2, n1/n3, n3/n5, n2/n4, n4/n6, n5/n6}
        \draw[->,>=stealth] (\from) -- (\to);
        \end{tikzpicture}
        \caption{Отношение \((X, R)\)}
        \label{fig:hasse}
    \end{figure}
    
    \begin{figure}[h!]
        \label{fig:hasse_full}
        \centering
        \begin{tikzpicture}
        [scale=.3,auto=left,every node/.style={circle,fill=blue!20}]
        \node (n1) at (4,10) {a};
        \node (n2) at (1,7)  {b};
        \node (n3) at (7,7)  {c};
        \node (n4) at (1,4) {d};
        \node (n5) at (7,4)  {e};
        \node (n6) at (4,1)  {f};
        \foreach \from/\to in {n1/n2, n1/n3, n3/n5, n1/n5, n2/n4, n1/n4, n4/n6, n5/n6, n2/n6, n3/n6, n1/n6}
        \draw[->,>=stealth] (\from) -- (\to);
        \end{tikzpicture}
        \caption{Замыкание \((X, R)\)}
    \end{figure}
    
    \begin{remark}
        Заметим, что на диаграмме не указаны петли --- их наличие и означает рефлексивность. 
    \end{remark}
    
    \textbf{Построение рефлексивного транзитивного замыкания:} 
    
    Положим сначала \(S = R\). После этого возьмем все пары \((aRb, bRc)\) и добавим \((a, c)\) в \(S\). Будем повторять этот процесс до тех пор, пока \(S\) не станет транзитивным отношением. 
    
    С другой стороны \(S\) --- пересечение всех отношений предпорядка \(\overline{R}\) на \(X\) таких что \(R \subset \overline{R}\).


\end{document}
