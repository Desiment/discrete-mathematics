\documentclass{article}
\usepackage{iftex}

%% Разрешить компиляцию только с движком LuaTex
\ifLuaTeX
\else
    \newlinechar 64\relax
    \errorcontextlines -1\relax
    \immediate\write20{@
        ************************************************@
        * LuaLaTex is required to compile this document.@
        * Sorry!@
        ************************************************}%
    \batchmode\read -1 to \@tempa
\fi

%% Для русификации достаточно подключить пакет fontspec и
%% выбрать Unicode шрифт в котором есть кириллические глифы. Ниже
%% основным шрифтом выбирается Unicode версия шрифта Computer Modern с заcечками
\usepackage{fontspec}
\setmainfont{CMU Serif}
\setsansfont{CMU Sans Serif}
\setmonofont{CMU Typewriter Text}

%% В XeLaTex или LuaLaTeX альтернативой известного пакета babel является пакет polyglossia.
%% Теперь у нас будут переносы слов
\usepackage{polyglossia}
\setdefaultlanguage{russian}
\setotherlanguage{english}

\usepackage[autostyle]{csquotes} % Правильные кавычки в зависимости от языка
\usepackage{totcount}
\usepackage{setspace}

% ToDo:
% [] MathNote   : \sum[] syntax
% [] MathNote   : \tikzmatrix 
% [] Layout     : A4 geometry
% [] Layout     : Define colors
% [] References : Load knowledge
% [] References : Create simple clever refs
% [] Fix hidding enviroment

%Отключить предупреждения об кастномной использовании пакетов "You have requested package..."
\usepackage{silence}
\WarningFilter{latex}{You have requested package}

\usepackage{xparse}
\usepackage{configuration/floats}
\usepackage{configuration/layout}
\usepackage{configuration/references}
\usepackage{configuration/mathnote}

\sloppy
% Окружения для набора задач
\newcounter{boxlblcounter}  

\newcommand{\task}[1]{\fbox{\begin{minipage}{4em}\centering\it #1\end{minipage}}}
\newcommand{\makeboxlabel}[1]{\fbox{\begin{minipage}{2em}\centering\it #1\end{minipage}}\hfill}% \hfill fills the label box
\newenvironment{tasklist}
  {\begin{list}
    {\arabic{boxlblcounter}}
    {\usecounter{boxlblcounter}
     \setlength{\labelwidth}{3em}
     \setlength{\labelsep}{0em}
     \setlength{\itemsep}{2pt}
     \setlength{\leftmargin}{1.5cm}
     \setlength{\rightmargin}{2cm}
     \setlength{\itemindent}{0em} 
     \let\makelabel=\makeboxlabel
    }
  }{\end{list}}


\newboolean{ShowHint}
\newboolean{ShowSolution}

% Подсказка
\newcommand{\hint}[1]{\ifthenelse{\boolean{ShowHint}}{\noindent\rotatebox[origin=c]{180}{\noindent
\begin{minipage}[t]{\linewidth} \noindent \it Указание: #1 \end{minipage}}}{}}

% solution: окружение для набора решений
% solution*: форсировано показывает решение, независимо от флага
\makeatletter
\newenvironment{solution*}[1][\text{Решение}]{ 
  \par
  \pushQED{\qed}%
  \normalfont
  \topsep0pt \partopsep3pt
  \trivlist
  \item[\hskip\labelsep
    \itshape
    #1\@addpunct{.}]\ignorespaces
}{
  \ifthenelse{\boolean{ShowSolution}}{
    \popQED\endtrivlist\@endpefalse
    \addvspace{6pt plus 6pt} % some space after
  }{\end{hidden}}
}
\makeatother

%https://tex.stackexchange.com/questions/533218/hiding-an-environment-that-contains-minted-code
%https://tex.stackexchange.com/questions/38150/in-lualatex-how-do-i-pass-the-content-of-an-environment-to-lua-verbatim
\RequirePackage{luacode}
\begin{luacode*}
do 
    function eat_buffer(buf)
        i,j = string.find(buf,"\\end{solution}")
        if i==nil then return "" else return nil end
    end
    function start_proccesing_solution(eat)
        if eat then luatexbase.add_to_callback('process_input_buffer', eat_buffer, 'eat_buffer') end
    end
    function stop_proccesing_solution(eat)
        if eat then luatexbase.remove_from_callback('process_input_buffer', 'eat_buffer') end
    end
end
\end{luacode*}
%https://tex.stackexchange.com/questions/537219/conditionals-inside-newcommand-with-empty-argument
%https://tex.stackexchange.com/questions/63223/xparse-empty-arguments
\ExplSyntaxOn
\DeclareExpandableDocumentCommand{\IfNoValueOrEmptyTF}{mmm}{\IfNoValueTF{#1}{#2}{\tl_if_empty:nTF {#1} {#2} {#3}}}
\NewDocumentCommand{\DefaultName}{mm}{\IfNoValueOrEmptyTF{#1}{#2}{#1}}
\ExplSyntaxOff

% WARNING!
% Buggy: one have always have
% \begin{solution}{NAME}
%   ....
% \end{solution}
% EVEN if NAME is empty
% if one starts some text on the same line as \begin{solution} or \end{solution} it could'not be hidden
\newenvironment{solution}[1]%
{
  \ifthenelse{\boolean{ShowSolution}}
    {\begin{solution*}[\DefaultName{#1}{\textit{Решение}}]\directlua{flag_eat=false}}
    {\directlua{flag_eat=true}}
  \directlua{start_proccesing_solution(flag_eat)}
}
{
  \ifthenelse{\boolean{ShowSolution}}{\end{solution*}}{}
  \directlua{stop_proccesing_solution(flag_eat)}
}



\title{Понятие мощности множества. Четыре теоремы Кантора}
\author{Михаил Михайлов}
\date{\today}
\newcommand{\ie}{т{.}~e{.}}
\newcommand{\card}{\mathrm{card}}
\mmzset{memo dir = images/cache/\jobname}
\makeatletter
\def\renewtheorem#1{%
  \expandafter\let\csname#1\endcsname\relax
  \expandafter\let\csname c@#1\endcsname\relax
  \gdef\renewtheorem@envname{#1}
  \renewtheorem@secpar
}
\def\renewtheorem@secpar{\@ifnextchar[{\renewtheorem@numberedlike}{\renewtheorem@nonumberedlike}}
\def\renewtheorem@numberedlike[#1]#2{\newtheorem{\renewtheorem@envname}[#1]{#2}}
\def\renewtheorem@nonumberedlike#1{  
\def\renewtheorem@caption{#1}
\edef\renewtheorem@nowithin{\noexpand\newtheorem{\renewtheorem@envname}{\renewtheorem@caption}}
\renewtheorem@thirdpar
}
\def\renewtheorem@thirdpar{\@ifnextchar[{\renewtheorem@within}{\renewtheorem@nowithin}}
\def\renewtheorem@within[#1]{\renewtheorem@nowithin[#1]}
\makeatother

\renewtheorem{theorem}{Теорема}[section]
\newtheorem*{defnotation}{Нотация}
\newtheorem{axiom}{Аксиома}


%Logic
\newcommand{\lxor}{\oplus}
\newcommand{\excludeor}{\bowtie}
%Set Theory
\newcommand{\class}[1]{\set{#1}}
\newcommand{\UniversalClass}{\mathfrak{V}}
\newcommand{\InductiveClass}{\mathrm{Ind}}

\newcommand{\Union}{\bigcup}
\newcommand{\Intersect}{\bigcap}

% Функции:
\newcommand{\MapTo}[3]{#1\colon #2 \to #3}
\newcommand{\setdiff}{\Delta}
\newcommand{\powerset}{\mathcal{P}}
\begin{document}

\maketitle
\tableofcontents


\subsubsection{Язык теории множеств}
\textbf{Формулы} в теории множеств строятся с помощью двух \textit{атомарных формул}:
\begin{itemize}
    \item \(x = y\)
    \item \(x \in y\)
\end{itemize}

И их \textbf{связок}:
\begin{itemize}
    \item \(\phi \land \psi\) --- "и"\,
    \item \(\phi \lor \psi\) --- "или"\,
    \item \(\phi \Rightarrow \psi\) --- "влечёт"\,
    \item \(\phi \Leftrightarrow \psi\) --- "равносильно"\,
    \item \(\neg\phi\) --- "отрицание"\,
\end{itemize}

А так же "кванторов"\,:
\begin{itemize}
    \item \(\forall x: \ldots\) — "для всех $x$ выполнено ..."
    \item \(\exists x: \ldots\) --- "существует $x$ такой, что ..."\,
\end{itemize}

\begin{remark}
    \label{rem:quantifier-relations-basic-operations}
    Кванторы можно интерпретировать как "бесконечные формулы"\,:
    \begin{align*}
        &\forall x \rightsquigarrow x_1 \land x_2 \land \ldots \\
        &\exists x \rightsquigarrow x_1 \lor  x_2 \lor  \ldots
    \end{align*}
\end{remark}

\subsubsection{Классы и их базовые свойства}
В этой секции \(\Phi(x)\) условие (возможно с параметрами). 
\begin{definition}
    \label{def:class}
    Класс \(C\) есть все  \(x\) такие что \(\Phi(x)\).
    \[C = \class{x : \Phi(x)}\]
\end{definition} 

Мотивация к понятию класса довольно очевидна: классы похожи на множества и поэтому с ними оперировать приятнее чем с формулами
\begin{defnotation}
    \(x \in C \coloneq \Phi(x)\)
\end{defnotation}

\begin{definition}
    \label{def:class-equality}
    Два класса \(C = \class{x : \Phi(x)}\) и \(D = \class{x : \Psi(x)}\) \textbf{равны} если \(\Phi(x) \Leftrightarrow \Psi(x)\)
\end{definition}
\begin{defnotation}
    \(C = D \coloneq \Phi(x) \Leftrightarrow \Psi(x)\)
\end{defnotation}

\begin{definition}
    \label{def:universal-class}
    \textbf{Универсальным классом} называется класс всех множеств.
\end{definition}
\begin{defnotation}
    \(\UniversalClass \coloneq \class{x: x = x}\)
\end{defnotation}

\begin{definition}
    \label{def:subclasses}
    Класс \(C = \class{x : \Phi(x)}\) \textbf{включен} в класс \(D = \class{x : \Psi(x)}\) если \(\Phi(x) \Rightarrow \Psi(x)\). 
    \begin{itemize}
        \item Если \(C\) включен в \(D\), то \(C\) \textbf{подкласс} \(D\).
        \item Если \(C\) включен в \(D\) и \(D\) не включен в \(C\), то  \(C\) \textbf{собственный подкласс} \(D\).
    \end{itemize}
\end{definition}
\begin{defnotation} \  
    \begin{itemize}
        \item \(C\) подкласс \(D\): \(C \subseteq D := \Phi(x) \Rightarrow \Psi(x)\).
        \item \(C\) собственный подкласс \(D\): \(C \subset D := \left(\Phi(x) \Rightarrow \Psi(x)\right) \land \left(\Psi(x) \not\Rightarrow \Phi(x)\right)\).
    \end{itemize}
\end{defnotation}

\begin{definition}
    \label{def:class-operations}
    Рассмотрим два класса \(C = \class{x : \Phi(x)}\) и \(D = \class{x : \Psi(x)}\) 
    \begin{itemize}
        \item Их \textbf{пересечением} называется класс \(C \cap D \coloneq \class{x: \Phi(x) \land \Psi(x)}\).
        \item Их \textbf{объединением} называется класс \(C \cup D \coloneq \class{x: \Phi(x) \lor \Psi(x)}\).
        \item Их \textbf{разностью} называется класс \(C \setminus D \coloneq \class{x: \Phi(x) \land \neg\Psi(x)}\)
        \item \textbf{Объединением} класса \(C\) называется класс \(\bigcup C \coloneq \class{x: x \in S\ \text{для некоторого}\ S \in C}\)
    \end{itemize}
\end{definition}

\begin{definition}
    \label{def:proper-class}
     Класс, не являющийся множеством, называется \textbf{собственным}.
\end{definition}

\begin{remark}
    \label{rem:set-as-class}
    Каждое множество \(S\) можно рассматривать как класс задаваемый формулой \(x \in S\). 
    
    Обратное, впрочем не верно. Для того что показать это, рассмотрим класс таких множеств, что не содержатся в себе: \(S = \set{x: x \not\in x}\). Если бы \(S\) было бы множеством, то было бы возможно два варианта:
    \begin{itemize}
        \item \(S \in S \Rightarrow S \not\in S\)
        \item \(S \not\in S \Rightarrow S \in S\)
    \end{itemize}
    В обоих случаях получаем противоречие, а значит \(S\) --- собственный класс.
\end{remark}

\begin{remark}
    \label{def:set-class-operations-symbols}
    Несмотря на то что по отношению к классам и множествам используются одинаковые символы, эти символы несут немного разный смысл, так как у них разная область определения. Таким образом, формально запись \(x \cap C\), где \(x\) --- множество, а \(C\) класс, некорректна. Однако, в связи с замечанием \ref{rem:set-as-class}, мы можем считать каждое множество классом, и тогда такие "смешанные"\, записи можно интерпретировать как записи над классами.
\end{remark}



\subsubsection{Система ZFC}
    \begin{defnotation}
        \(\varnothing \coloneq \set{x : x \ne x}\)
    \end{defnotation}
    \begin{definition}
        \label{def:inductive-class}
        Множество \(X\) называется \textbf{индуктивным} если:
        \[\varnothing \in X \land \forall u: u \in X \Rightarrow \{u, \{u\}\} \in X\]
        
        Класс индуктивных множеств: \(\InductiveClass\)
    \end{definition}

    \begin{axiom}[Аксиома экстенсиональности]
        Два множества равны если их элементы совпадают:
        \begin{equation}
            \label{ZFC-Ext}
            \forall X: \forall Y: \forall u: \left(u \in X \Leftrightarrow u \in Y \right) \Rightarrow X = Y \tag{Ext}
        \end{equation}
    \end{axiom}
    \begin{axiom}[Аксиома пустого множества]
        Существует пустое множество:
        \begin{equation}
            \label{ZFC-Empty}
            \exists X: \forall u: u \in X \Leftrightarrow u \ne u \tag{Empty}
        \end{equation}
    \end{axiom}
    \begin{axiom}[Аксиома пары]
        Для всех \(a, b\) существует множество состоящее в точности из \(a\) и \(b\):
        \begin{equation}
            \label{ZFC-Pair}
            \forall X,Y: \exists Z: \forall u: u \in Z \Leftrightarrow (u = X \lor  u = Y) \tag{Pair}
        \end{equation}
    \end{axiom}
    \begin{axiom}[Схема аксиом выделения]
        Для каждого условия \(\Phi(x)\) (возможно с параметрами) имеется своя аксиома выделения:
        \begin{equation}
            \label{ZFC-Sep}
            \forall X: \exists Y: \forall u: u \in Y \Leftrightarrow (u \in X \land \Phi(u)) \tag{Sep}
        \end{equation}
    \end{axiom}
    \begin{axiom}[Аксиома объединения]
        Для любого множества \(X\) существует объединение его элементов:
        \begin{equation}
            \label{ZFC-Union}
            \forall X: \exists Y: \forall u: u \in Y \Leftrightarrow \exists v: (v \in X \land u \in v) \tag{Union}
        \end{equation}
    \end{axiom}
    \begin{axiom}[Аксиома степени]
        Для любого множества \(X\) существует множество  всех подмножеств \(X\):
        \begin{equation}
            \label{ZFC-Power}
            \forall X: \exists Y: \forall u: u \in Y \Leftrightarrow u \subseteq X \tag{Power}
        \end{equation}
    \end{axiom}
    \begin{axiom}[Аксиома бесконечности]
        Существует индуктивное множество:
        \begin{equation}
            \label{ZFC-Inf}
            \exists X \in \InductiveClass \tag{Inf}
        \end{equation}
    \end{axiom}
    \begin{axiom}[Схема аксиом подстановки]
        Пусть \(\Phi(x,y)\) — инъективное по \(y\) условие (возможно с параметрами). Тогда для любого множества \(X\) найдется множество \(Y\) что для всякого \(y \in Y\) найдется \(x \in X\), такой что \(\Phi(x, y)\)
        \begin{equation}
            \label{ZFC-Repl}
            \left(\forall x, y_1, y_2: \Phi(x_1, y) \land \Phi(x_2, y) \Rightarrow y_1 = y_2 \right) \Rightarrow  \forall X:\exists Y: \forall y: (y \in Y \Leftrightarrow \exists x: x \in X \land \Phi(x, y))) \tag{Repl}
        \end{equation}
    \end{axiom}
    \begin{axiom}[Аксиома регулярности]
        Всякое непустое множество содержит элемент, не содержащий в себе подмножество исходного множества.
        \begin{equation}
            \label{ZFC-Reg}
            \forall X: X \ne \varnothing \Rightarrow (\exists u: u \in X \land X \cap u = \varnothing) \tag{Reg}
        \end{equation}
    \end{axiom}
    \begin{axiom}[Аксиома выбора]
        Для всякого семейства \(X\) непустых множеств существует функция \(f\), которая каждому множеству семейства сопоставляет один из элементов этого множества
        \begin{equation}
            \label{ZFC-Choice}
            \forall X: \varnothing \not\in X \Rightarrow \left(\exists f: f: X \to \bigcup X \land \forall u \in X: f(u) \in u\right) \tag{Choice}
        \end{equation}
    \end{axiom}

\subsubsection{Замечания к аксиомам}

    \begin{remark}
        \label{rem:zfc-ext}
        Аксиома \ref{ZFC-Ext} выражает фундаментальную идею множества: каждое множество определяется своими элементами.
    \end{remark}

    \begin{remark}
        \label{rem:zfc-empty}
        Очевидно, что пустое множество единственно в силу аксиомы \ref{ZFC-Ext}. Как будет видно далее, на самом деле аксиома пустого множества не является "обязательной"\,: для того чтобы получить пустое множество надо взять любое множество и применить к нему схему аксиом выделения \ref{ZFC-Sep} с формулой \(x \ne x\). Однако существование хотя бы одного множества надо постулировать.
    \end{remark}

    \begin{remark}
        \label{rem:sets-properties}
        Основные свойства множеств:
        \begin{itemize}
            \item Либо \(x \in y\), либо \(y \in x\), либо ни то, ни другое. В частности, всегда верно \(x \not\in x\). 
            \item \(x_1 \in x_2 \in \ldots \in x_n \Rightarrow x_n \not\in x_1\).
        \end{itemize}
    \end{remark}
    \begin{proof} \ 

        \fbox{1} Предположим, что \(x \in y\) и \(y \in x\). Рассмотрим \(\set{x, y}\). По аксиоме \ref{ZFC-Reg} имеем что либо \(x \cap \set{x, y} = \varnothing\), либо \(y \cap \set{x, y} = \varnothing\), либо и то, и то.
        
        Но при этом, так как \(y \in x\) имеем \(x \cap \set{x, y} \ne \varnothing\), т.к. пересечение содержит хотя бы \(y\). Аналогично \(y \cap \set{x, y} \ne \varnothing\), противоречие. 
        
        \fbox{2} Предположим, что \(x_n \in x_1\). Тогда \(\forall x_i \in A = \set{x_1, \ldots, x_n}: x_i \cap A \ne \varnothing\), противоречие с \ref{ZFC-Reg}.
    \end{proof}
    \begin{defnotation}
        Существует несколько фундаментальных способов получать новые множества из имеющихся:
        \begin{itemize}
            \item Перечислением элементов: \(x_1, x_2, \ldots x_n \rightsquigarrow \set{x_1, x_2, \ldots, x_n}\)
            \item Объединение двух множеств: \(X, Y \rightsquigarrow X \cup Y\)
            \item Выделением по предикату: \(U, \Phi(x) \rightsquigarrow \set{x \in U \,|\, \Phi(x)} \subset U\)
            \item Булеан, множество всех подмножеств: \(X \rightsquigarrow \powerset(X)\)
        \end{itemize}
    \end{defnotation}
    \begin{defnotation}
        С помощью объединения и выделения можно построить пересечение, разность и симметрическую разность множеств:
    \begin{align*}
        &X \cap Y \coloneq \set{x \in X \cup Y \,|\, x \in X \land x \in Y} \\
        &X \setminus Y \coloneq \set{x \in X \cup Y \,|\, x \in X \land x \not\in Y} \\
        &X \setdiff Y \coloneq \set{x \in X \cup Y \,|\, x \in X \lxor x \in Y}
    \end{align*}
    \end{defnotation}

\section{Основные теоретико-множественные конструкции}

\subsubsection{Аксиома пары. Упорядоченные пары}
    \begin{definition}
        \label{def:singleton}
        \textbf{Синглтон} \(\set{a} \coloneq \set{a, a}\)
    \end{definition}
    \begin{remark}
        \label{rem:singleton-properties}
        Cинглтон --- множество (следует из \ref{ZFC-Pair}).  Более того, из \ref{ZFC-Ext} следует: \(a = b \Leftrightarrow \set{a} = \set{b}\). Понятно также, что синглтон не равен своему содержанию, если это так, то каждый элемент \(\set{a}\) принадлежит \(a\), в частности \(a \in a\), что противоречит \ref{rem:sets-properties}.    
    \end{remark}

    \begin{definition}
        \label{def:ordered-pair}
        Упорядоченная пара --- множество: \((a, b) \coloneq \set{\set{a}, \set{a,b}}\).
    \end{definition}
    \begin{remark}
        \label{rem:pair-properties}
        Неформально говоря, упорядоченная пара — это множество, в котором один элемент отмечен как первый, а другой как второй. Существуют другие способы определить упорядоченную пару, например:
        \begin{align}
            &(a, b) \coloneq \set{a, \set{a, b}}\ \text{упрощенный вариант определения} \\
            &(a, b):= \set{\set{a},\set{b, \set{\varnothing}}}\ \text{по Куратовскому}
        \end{align}
        
        Из \ref{ZFC-Pair} следует, что упорядоченная пара --- множество.
    \end{remark}
    \begin{proposition}[Основное свойство упорядоченной пары]
        \label{prop:pair-main-property}
        \[(a, b) = (c,d) \Leftrightarrow a = c \land b = d\]
    \end{proposition}
    \begin{proof}
        \[(a, b) = (c, d) \Leftrightarrow \set{\set{a},\set{a, b}} = \set{\set{c},\set{c, d}} \Leftrightarrow a = c \land b = d\]

        Второй переход требует некоторых пояснений: из \ref{ZFC-Ext} следует, что либо \(\set{a} = \set{c, d}\), либо \(\set{a} = \set{c}\). 
        \begin{itemize}
            \item В первом случае $c = d = a = b$. 
            \item Во втором случае по замечанию \ref{rem:singleton-properties}: \(a = c\). Тогда, \(\set{a, b} = \set{c, d} = \set{a, d}\) и по \ref{ZFC-Ext} получаем что \(b = d\).
        \end{itemize}
    \end{proof}

    \begin{definition}
        \label{def:ordered-cortege}
        \textbf{Упорядоченной \(n\)-кой} называется множество:
        \begin{itemize}
            \item  \((a_1, a_2)\) при \(n=2\)
            \item \(((a_1, \ldots, a_{n-1}), a_n)\) при \(n > 2\)
        \end{itemize}
    \end{definition}
    \begin{remark}
        \label{rem:ordered-pair-universal-property}
        Очевидно, что определение упорядоченной \(n\)-ки не единственно. Например, можно брать вместо \((a_1, a_2, a_3) \coloneq ((a_1, a_2), a_3)\) другое определение упорядоченной тройки: \((a_1, a_2, a_3) \coloneq (a_1, (a_2, a_3))\). Однако, и то, и другое определение обладают основным свойством:
        \[(a_1, a_2, a_3, \ldots, a_n) = (b_1, b_2, b_3, \ldots, b_n) \Leftrightarrow \begin{cases} a_1 = b_1 \\ \ldots\\ a_n = b_n\end{cases}\]
    \end{remark}

\subsubsection{Булеан множества}

    TBD: Example of Boolean

    \begin{proposition}
        \label{prop:powerset-notsubset-set}
        Не существует такого множества \(X\), что \(\powerset(X) \subset X\).
    \end{proposition}
    \begin{proof}
        Пусть такое множество \(X\) существует.
        Мы знаем, что \(X \in \powerset(X) \subset X\). Это значит что \(X \in X\), что противоречит \ref{rem:sets-properties}.
    \end{proof}

    С помощью булеана можно получить все конечные множества:
    \[\varnothing;\ 2^\varnothing = \set{\varnothing}; 2^{\set{\varnothing}} = \set{\varnothing, \set{\varnothing}}; \ldots\]
    \begin{remark}
        Чтобы получать не только степени двойки, можно объединять разные множества: так например множество из 3-х элементов можно получить следующим образом 
        \[\set{\varnothing, \set{\varnothing}, \set{\varnothing, \set{\varnothing}}}\] 
    \end{remark}
    Введем обозначения
    \begin{align}
        &0 := \varnothing \\
        &1 := \set{\varnothing}\\
        &2 := \set{\varnothing, \set{\varnothing}} \\
        &\dots \\
        &n := \set{0, 1, \ldots, n - 1}
    \end{align}
    Тогда постулируется существование множества \(\mathbb N \eqdef \set{0, 1, 2, \ldots}\). 

\subsubsection{Декартово произведение множеств}
    У каждого множества \(X\) есть множество всех его подмножеств \(\powerset(X)\). 

    Заметим, что если \(a \in X,\, b \in Y\), то \((a, b) \in 2^{2^{X \cup Y}}\). Поэтому, используя \ref{ZFC-Sep}, получаем следующую конструкцию: 
    \begin{definition}
        \label{def:decart-product}
        Декартово произведение множеств \(X\) и \(Y\) это множество всех таких пар \((x, y)\) что \(x \in X\), \(y \in Y\). 
        
        Обозначается как: \(X \times Y \coloneq \{(a,b) \in 2^{2^{X \cup Y}}\,|\, a \in X \land b \in Y\}\)
    \end{definition}
    \begin{definition}
        \label{def:decart-product-of-n-sets}
        Аналогично можем определить декартово произведение \(n\) множеств
        \begin{enumerate}
            \item \(X_1 \times X_2 \times\ldots\times X_n \coloneq (X_1 \times X_2 \times\ldots\times X_{n-1})\times X_n\).
            \item \(X_1 \times X_2 \times\ldots\times X_n \coloneq \set{(x_1,\ldots,x_n) \in T_n\,|\, x_1 \in X_1 \land\ldots\land x_n \in X_n}\), где \(T_n = \powerset(\powerset(T_{n-1}\cup X_n))\), \(T_1 = X_1\)
        \end{enumerate}
    \end{definition}
    \begin{remark}
        \label{rem:equal-defs-of-decart-product}
        Определения \ref{def:decart-product-of-n-sets}.1 и  \ref{def:decart-product-of-n-sets}.2 равносильны так как в обоих случаях итоговое множество состоит из одних и тех же элементов.
    \end{remark}
    \begin{defnotation}
        \(X^n \coloneq \underset{n\ \text{раз}}{X \times\ldots\times X}\)
    \end{defnotation}

    TBD: Images, examples

    

\end{document}
