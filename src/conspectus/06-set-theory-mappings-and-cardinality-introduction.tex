\documentclass{article}
\usepackage{iftex}

%% Разрешить компиляцию только с движком LuaTex
\ifLuaTeX
\else
    \newlinechar 64\relax
    \errorcontextlines -1\relax
    \immediate\write20{@
        ************************************************@
        * LuaLaTex is required to compile this document.@
        * Sorry!@
        ************************************************}%
    \batchmode\read -1 to \@tempa
\fi

%% Для русификации достаточно подключить пакет fontspec и
%% выбрать Unicode шрифт в котором есть кириллические глифы. Ниже
%% основным шрифтом выбирается Unicode версия шрифта Computer Modern с заcечками
\usepackage{fontspec}
\setmainfont{CMU Serif}
\setsansfont{CMU Sans Serif}
\setmonofont{CMU Typewriter Text}

%% В XeLaTex или LuaLaTeX альтернативой известного пакета babel является пакет polyglossia.
%% Теперь у нас будут переносы слов
\usepackage{polyglossia}
\setdefaultlanguage{russian}
\setotherlanguage{english}

\usepackage[autostyle]{csquotes} % Правильные кавычки в зависимости от языка
\usepackage{totcount}
\usepackage{setspace}

% ToDo:
% [] MathNote   : \sum[] syntax
% [] MathNote   : \tikzmatrix 
% [] Layout     : A4 geometry
% [] Layout     : Define colors
% [] References : Load knowledge
% [] References : Create simple clever refs
% [] Fix hidding enviroment

%Отключить предупреждения об кастномной использовании пакетов "You have requested package..."
\usepackage{silence}
\WarningFilter{latex}{You have requested package}

\usepackage{xparse}
\usepackage{configuration/floats}
\usepackage{configuration/layout}
\usepackage{configuration/references}
\usepackage{configuration/mathnote}

\sloppy
% Окружения для набора задач
\newcounter{boxlblcounter}  

\newcommand{\task}[1]{\fbox{\begin{minipage}{4em}\centering\it #1\end{minipage}}}
\newcommand{\makeboxlabel}[1]{\fbox{\begin{minipage}{2em}\centering\it #1\end{minipage}}\hfill}% \hfill fills the label box
\newenvironment{tasklist}
  {\begin{list}
    {\arabic{boxlblcounter}}
    {\usecounter{boxlblcounter}
     \setlength{\labelwidth}{3em}
     \setlength{\labelsep}{0em}
     \setlength{\itemsep}{2pt}
     \setlength{\leftmargin}{1.5cm}
     \setlength{\rightmargin}{2cm}
     \setlength{\itemindent}{0em} 
     \let\makelabel=\makeboxlabel
    }
  }{\end{list}}


\newboolean{ShowHint}
\newboolean{ShowSolution}

% Подсказка
\newcommand{\hint}[1]{\ifthenelse{\boolean{ShowHint}}{\noindent\rotatebox[origin=c]{180}{\noindent
\begin{minipage}[t]{\linewidth} \noindent \it Указание: #1 \end{minipage}}}{}}

% solution: окружение для набора решений
% solution*: форсировано показывает решение, независимо от флага
\makeatletter
\newenvironment{solution*}[1][\text{Решение}]{ 
  \par
  \pushQED{\qed}%
  \normalfont
  \topsep0pt \partopsep3pt
  \trivlist
  \item[\hskip\labelsep
    \itshape
    #1\@addpunct{.}]\ignorespaces
}{
  \ifthenelse{\boolean{ShowSolution}}{
    \popQED\endtrivlist\@endpefalse
    \addvspace{6pt plus 6pt} % some space after
  }{\end{hidden}}
}
\makeatother

%https://tex.stackexchange.com/questions/533218/hiding-an-environment-that-contains-minted-code
%https://tex.stackexchange.com/questions/38150/in-lualatex-how-do-i-pass-the-content-of-an-environment-to-lua-verbatim
\RequirePackage{luacode}
\begin{luacode*}
do 
    function eat_buffer(buf)
        i,j = string.find(buf,"\\end{solution}")
        if i==nil then return "" else return nil end
    end
    function start_proccesing_solution(eat)
        if eat then luatexbase.add_to_callback('process_input_buffer', eat_buffer, 'eat_buffer') end
    end
    function stop_proccesing_solution(eat)
        if eat then luatexbase.remove_from_callback('process_input_buffer', 'eat_buffer') end
    end
end
\end{luacode*}
%https://tex.stackexchange.com/questions/537219/conditionals-inside-newcommand-with-empty-argument
%https://tex.stackexchange.com/questions/63223/xparse-empty-arguments
\ExplSyntaxOn
\DeclareExpandableDocumentCommand{\IfNoValueOrEmptyTF}{mmm}{\IfNoValueTF{#1}{#2}{\tl_if_empty:nTF {#1} {#2} {#3}}}
\NewDocumentCommand{\DefaultName}{mm}{\IfNoValueOrEmptyTF{#1}{#2}{#1}}
\ExplSyntaxOff

% WARNING!
% Buggy: one have always have
% \begin{solution}{NAME}
%   ....
% \end{solution}
% EVEN if NAME is empty
% if one starts some text on the same line as \begin{solution} or \end{solution} it could'not be hidden
\newenvironment{solution}[1]%
{
  \ifthenelse{\boolean{ShowSolution}}
    {\begin{solution*}[\DefaultName{#1}{\textit{Решение}}]\directlua{flag_eat=false}}
    {\directlua{flag_eat=true}}
  \directlua{start_proccesing_solution(flag_eat)}
}
{
  \ifthenelse{\boolean{ShowSolution}}{\end{solution*}}{}
  \directlua{stop_proccesing_solution(flag_eat)}
}



\title{Понятие мощности множества. Три теоремы Кантора}
\author{Михаил Михайлов}
\date{\today}
\newcommand{\ie}{т{.}~e{.}}
\DeclareMathOperator{\card}{card}
\mmzset{memo dir = images/cache/\jobname}
\makeatletter
\def\renewtheorem#1{%
  \expandafter\let\csname#1\endcsname\relax
  \expandafter\let\csname c@#1\endcsname\relax
  \gdef\renewtheorem@envname{#1}
  \renewtheorem@secpar
}
\def\renewtheorem@secpar{\@ifnextchar[{\renewtheorem@numberedlike}{\renewtheorem@nonumberedlike}}
\def\renewtheorem@numberedlike[#1]#2{\newtheorem{\renewtheorem@envname}[#1]{#2}}
\def\renewtheorem@nonumberedlike#1{  
\def\renewtheorem@caption{#1}
\edef\renewtheorem@nowithin{\noexpand\newtheorem{\renewtheorem@envname}{\renewtheorem@caption}}
\renewtheorem@thirdpar
}
\def\renewtheorem@thirdpar{\@ifnextchar[{\renewtheorem@within}{\renewtheorem@nowithin}}
\def\renewtheorem@within[#1]{\renewtheorem@nowithin[#1]}
\makeatother

\renewtheorem{theorem}{Теорема}[section]
\DeclareMathOperator{\id}{id}

\begin{document}

\maketitle
\tableofcontents

\newpage
\section{Функции}

\begin{definition}
    \label{def:function-restriction}
    \textbf{Cужение функции} \(f: X \rightarrow Y\) на множество \(A \subset X\) --- функция \(f\,|_A: A \rightarrow Y\) совпадающая с \(f\) на всех элементах \(A\).  
\end{definition}

\begin{definition}
    \label{def:function-graphics}
    \textbf{Графиком функции} \(f: X \rightarrow Y\) называется множество \(\Gamma_f = \set{(x, f(x)) | x \in X}\).
\end{definition}


\subsection{Образ, прообраз и их свойства}
\begin{definition}
    \label{def:function-image-preimage}
    Пусть \(f: X \rightarrow Y\) отображение.
    \begin{itemize}
        \item \textbf{Образ множества \(A \subset X\)} называется множество \(f(A) = \set{f(a) | a \in A}\);
        \item \textbf{Прообразом множества \(B \subset Y\)} называется множество \(f^{-1}(B) = \set{x \in X | f(x) \in B}\);
    \end{itemize}
\end{definition}


\begin{proposition}[Свойства образа]
    Пусть \(f: X \rightarrow Y\). Тогда 
    \begin{itemize}
        \item \(\forall A, B \subset X\): \(f(A \cup B) = f(A) \cup f(B)\)
        \item \(\forall \set{A_i}_{i \in I},\, A_i \subset X\):  \(f(\bigcup_{i \in I} A_i) = \bigcup_{i \in I} f(A_i)\)
    \end{itemize}
    Так же, \(f\) инъективно тогда и только тогда, когда:  \(\forall A, B \subset X\): \(f(A \cap B) = f(A) \cap f(B)\)
\end{proposition}
\begin{proof}
    Тривиально (используется прямое вычисление и определения)
\end{proof}

\begin{proposition}[Свойства прообраза]
    Пусть \(f: X \rightarrow Y\). Тогда 
    \begin{itemize}
        \item \(\forall A, B \subset Y\): \(f^{-1}(A \cup B) = f^{-1}(A) \cup f^{-1}(B)\)
        \item \(\forall A, B \subset Y\): \(f^{-1}(A \cap B) = f^{-1}(A) \cap f^{-1}(B)\)
        \item \(\forall A, B \subset Y\): \(f^{-1}(A \setminus B) = f^{-1}(A) \setminus f^{-1}(B)\)
        \item \(\forall A, B \subset Y\): \(f^{-1}(A \Delta B) = f^{-1}(A) \Delta f^{-1}(B)\)
        \item \(\forall \set{A_i}_{i \in I},\, A_i \subset Y\):  \(f^{-1}(\bigcup_{i \in I} A_i) = \bigcup_{i \in I} f^{-1}(A_i)\)
        \item \(\forall \set{A_i}_{i \in  I},\, A_i \subset Y\):  \(f(\bigcap_{i \in I} A_i) = \bigcap_{i \in  I} f(A_i)\)
    \end{itemize}
\end{proposition}
\begin{proof}
    Тривиально (используется прямое вычисление и определения).
\end{proof}

\begin{remark}
    Если \(A\) одноточечно, \(A = \set{x_0}\), часто пишут \(f(x_0)\) вместо \(f(\set{x_0})\). Аналогично для прообраза.  
\end{remark}

\subsection{Инъективные, сюръективные и биективные отображения}
\begin{definition}
    \label{def:function-injection-surjection-bijection}
    Пусть \(f: X \rightarrow Y\) отображение:
    \begin{itemize}
        \item Говорят что \(f\) --- \textbf{инъекция}, если: 
        \[\forall\, a_1, a_2 \in X: f(a_1) = f(a_2) \Rightarrow a_1 = a_2\]
        \item Говорят что  \(f\) --- \textbf{сюръекция} если
        \[\forall\, b \in Y: f^{-1}(b) \neq \varnothing\]
        \item Говорят что  \(f\) --- \textbf{биекция} если \(f\) оно сюръекция и инъекция одновременно.
    \end{itemize}
\end{definition}

\subsection{Композиция функций и её свойства}

\begin{definition}
    \label{def:composition}
    Пусть \(f: X \rightarrow Y\) и \(g: Y \rightarrow Z\), то \textbf{композиция} \(h = g \circ f\) есть отображение \(h(x) = g(f(x))\).
    
    Композиция считается \textbf{полностью заданной}, если \(D(g) \subset E(f)\). 
    
    \textbf{Частичная композиция} функций \(f: X \rightarrow Y\) и \(g: Y_1 \rightarrow Z\), где \(Y_1 \subset Y\).
\end{definition}

\begin{proposition}
    \label{prop:comp-properties}
    Свойства полной композиции. Пусть \(f: X \rightarrow Y, g: Y \rightarrow Z\).
    \begin{enumerate}
        \item Если \(f, g\) --- инъективны, то \(g \circ f\) --- иънективно.
        \item Если \(f, g\) --- cрюъективны, то \(g \circ f\) --- сюръективно.
        \item Если \(g \circ f\) --- сюръективна, то \(g\) --- сюръективна.
        \item Если \(g \circ f\) --- инъективно, то \(f\) --- инъективна.
        \item \((g \circ f) (A) = g(f(A)), \forall\, A \subset X\).
        \item \((g \circ f)^{-1}(A) = f^{-1}(g^{-1}(A)), \forall\, A \subset Z\)
        \item Если \(f, g\) обратимы, то \(g \circ f\) обратима. 
    \end{enumerate}
\end{proposition}
\begin{proof} Докажем каждое свойство: 

    \textbf{Свойство 1}: По определению. 
    
    \textbf{Свойство 2}: По определению. 
    
    \textbf{Свойство 3}: Если \(g\) не сюръективна, то существует \(z \in Z\) такой, что \(g^{-1}(z) = \varnothing\). Но так как \(g \circ f\) сюръективна, для\(z\) существует по крайне мере один \(x\) такой что \(g(f(x)) = z\). Тогда \(f(x) \in g^{-1}(z)\). Противоречие.
    
    \textbf{Свойство 4}: Если \(f\) не инъективна, то \(\exists x_1 \ne x_2: f(x_1) = f(x_2)\), но тогда \(g \circ f (x_1) = g \circ f (x_2)\), т.е. \(g \circ f\) не инъективно. Противоречие. 
    
    \textbf{Свойство 5}: По определению. 
    
    \textbf{Свойство 6}: В силу свойства 5 имеем \((g \circ f)(A) = g(f(A)) \Rightarrow (g \circ f)^{-1}(Z) = f^{-1} \circ g^{-1} = f^{-1}\mleft(g^{-1}(Z)\mright)\)
    
    \textbf{Свойство 7}: Известно, что функция обратима тогда и только тогда, когда она инъективна (см предложение \ref{prop:left-inverse-injection}), а значит свойство равносильно свойству 1.
\end{proof}


\subsection{Обратные отображения}
\begin{definition}
    \label{def:left-inverse-function}
    Если есть инъекция \(f: X \rightarrow Y\), то функция \(g: f(X) \rightarrow X\) такая что \(g(f(x)) = x\) называется функцией, \textbf{левой обратной} к \(f\).
\end{definition}

\begin{proposition}
    \label{prop:left-inverse-injection}
    У функция \(f\) есть левая обратная тогда и только тогда, когда она инъективна. 
\end{proposition}
\begin{proof}
    Ко всякой инъективной функции уже была построена обратная выше.
    Теперь предположим, что у \(f\) есть обратная функция, но при этом \(f\) не инъективна: т.е. существуют \(x_1 \ne x_2: f(x_1) = f(x_2)\). Тогда значение обратной функции \(g\) в точке \(f(x_1)\) не определено. Противоречие. 
\end{proof}

\newpage

\section{Мощность множеств}

Нформально, мощность множества это такое обобщение идеи \textit{числа}, чтобы говорить о размере множества, то есть о количестве элементов в нем. Формально это некоторое расширение множества натуральных чисел, которое позволяет сравнивать между собой \enquote{разные} бесконечности.


Для конечных множеств мощность это просто количество элементов в множестве, например, множество \(\set{a, b}\) имеет мощность 2. Но как быть с бесконечными множествами? Тут уже количество элементов не посчитать обычным способом. Георг Кантор занимался теорией множеств и ввел понятие мощности. Идея, которую использовал Кантор, была следующая: если между двумя множествами есть биекция (взаимно-однозначное соответствие), то эти множества должны иметь одинаковую мощность. Например, натуральные числа и чётные числа: каждому натуральному \(n\) можно поставить в соответствие \(2n\), и такое отображение будет биекцией. Значит, их мощности равны, хотя интуитивно кажется, что чётных чисел меньше.

На первый взгляд возникает \enquote{парадокс}: часть множества равна по мощности целому. Однако, это просто особенность бесконечных множеств. Кантор, показал, что существуют разные \enquote{уровни} бесконечности. Например, множество действительных чисел имеет большую мощность, чем натуральные: между этим двумя множествами нет биекции. До Кантора математики не очень формально подходили к бесконечностям. Кантор систематизировал понятие бесконечности, ввёл сравнение через биекции, показал, что бесконечности бывают разными. Это, вызвало споры, но в итоге стало основой для теории множеств и современной математики.

Мы не будем пока вводить формально мощность как объект, расширяющий понятие числа, а будем думать про мощность в контексте двух множеств и пытаться их сравнивать между собой.

\begin{definition}
    \label{def:set-cardinality-equal}
    Мощность \(A\) \textbf{равна} мощности \(B\) тогда и только тогда, когда  существует биекция между \(A\) и \(B\).
    \[
        \card A = \card B \overset{\text{\tiny def}}{\Leftrightarrow} \exists f: A \rightarrow B, \text{где \(f\) --- биекция}
    \]
\end{definition}
\begin{proposition}
    Множество натуральных чисел равномощно множеству четных натуральных чисел: 
    \[
        \card \N = \card \N_{\text{чет}}
    \]
\end{proposition}
\begin{proof}
    Рассмотрим отображение 
    \begin{align*}
        & f \colon \N \to \N_{\text{чет}} \\
        &n \mapsto 2n
    \end{align*}
    и проверим что оно является биекцией. Для этого рассмотрим отображение:
    \begin{align*}
        & g \colon \N_{\text{чет}} \to N \\
        & n \mapsto \frac{n}{2}
    \end{align*}
    Легко видеть что \(f \circ g = \id_{\N_{\text{чет}}}\) и \(g \circ f = \id_{\N}\), а значит \(g\) это обратная к \(f\), поэтому \(f\) биекция (по предложению TODO)
\end{proof}

\begin{definition}
    \label{def:set-cardinality-order}
    Мощность \(A\) \textbf{не больше} мощности \(B\) тогда и только тогда, когда существует инъекция из \(A\) в \(B\). 
    \[\card A \leqslant \card B \overset{\text{\tiny def}}{\Leftrightarrow} \exists f: A \rightarrow B, \text{где \(f\) --- инъекция}\]
\end{definition}
\begin{remark}
    В частности если \(A \subseteq B\), то существует инъекция \(f \colon A \injto B\) которая просто отображает \(A\) в себя: \(f(a) = a\), поэтому мощность подмножества не может быть строго больше мощности множества.
\end{remark}
\begin{remark}
    Разумно ожидать что если \(\card A \leq \card B\) и \(\card B \leq \card A\), то \(\card A = \card B\). Это действительно так, однако, это совсем нетривиальный результат.
\end{remark}

\begin{proposition}
    \label{prop:set-cardinality-order}
    Мощность \(A\) не больше мощности \(B\), тогда и только тогда, когда существует сюръекция из \(B\) в \(A\).  
    \[\card A \leqslant \card B \Leftrightarrow \exists f: B \rightarrow A, \text{где \(f\) --- сюръекция}\]
\end{proposition}
\begin{proof} 
    Докажем в обе стороны (\(A\), \(B\) не пусты): 

    \fbox{\(\Leftarrow\)} Если существует сюръекция \(f: B \rightarrow A\), то чтобы получить инъекцию из \(A\) в \(B\) достаточно выбрать из каждого прообраза каждой точки \(A\) по одному элементу \(B\).   
    
    \fbox{\(\Rightarrow\)} Если \(\card A \leqslant \card B\), то существует инъекция \(g: A \rightarrow B\), тогда для всех точек \(x \in g(A) \subset B\) положим \(f(x) = g^{-1}(x)\), а для \(y \in B \setminus g(A)\) положим \(f(y) = a \in A\).  
\end{proof}

\begin{definition}
    \label{def:countable-set}
    Если \(X\) равномощно \(\N\), \(X\) называется \textbf{счетным}.
\end{definition}

\begin{definition}
    \label{def:continuum}
    Если \(X\) равномощно \([0; 1]\) то \(X\) имеет мощность \textbf{континуума}. 
\end{definition}

\subsection{Бесконечные множества}
\begin{proposition}
    \label{prop:basic-bijection}
    \[\card (0; 1) \underset{1}{=} \card (0; 1] \underset{2}{=} \card [0; 1) \underset{3}{=} \card [0; 1]\]
\end{proposition}
\begin{proof}
    Равенство \((2)\) очевидно, биекцией будет:
    \[f(x) = \begin{cases}x,\, x \in (0; 1), \\ 1,\, x = 0 \\ 0,\, x = 1 \end{cases}\]
    
    Чтобы установить равенство \((1)\) и \((3)\) воспользуемся следующим отображением
    \[f(x) = \begin{cases} \frac{x}{2},\, x \in \set{\frac{1}{2^n}, n \in \mathbb N_0} \\ x,\,\ \text{иначе}\end{cases}\]
    
    Понятно, что \(f\) обратима:
    \[f^{-1}(x) = \begin{cases} 2x,\, x \in \set{\frac{1}{2^n}, n \in \mathbb N} \\ x,\,\ \text{иначе}\end{cases}\]
    
    Получаем: 
    \begin{align*}
        f\mleft((0; 1]\mright) &\underset{1}{=} (0; 1) \\
        f\mleft([0; 1]\mright) &\underset{3}{=}  [0; 1)
    \end{align*}
\end{proof}
    
\begin{corollary}
    \label{cor:real-numb-colloc}
    \(\R\) имеет мощность континуума: отображение:
    \begin{align*}
        f(x): (0; 1) \rightarrow \R \\
        x \mapsto \tan\mleft(\pi x + \frac{\pi}{2}\mright)
    \end{align*}
    будет биекцией.
\end{corollary}

\subsection{Теорема Кантора о неравномощности отрезка и натурального ряда}
\begin{theorem}[Первая теорема Кантора]
    \label{th:first-cantor-th}
    \[\card \N \neq \card [0;1]\]
\end{theorem}
\begin{proof}
    Пусть существует биекция \(f: \N \rightarrow [0;1]\). Построим бесконечную десятичную дробь \[x = 0,x_1x_2x_3\ldots\] следующим образом: 
    \begin{itemize}
        \item Если первая цифра после запятой у \(f(1)\) это не 5, то положим \(x_1 = 5\). Иначе положим \(x_1 = 4\).
        \item Если вторая цифра после запятой у \(f(2)\) это не 5, то положим \(x_2 = 5\). Иначе положим \(x_2 = 4\).
        \item \(\ldots\)
        \item Если \(n\)-ая цифра после запятой у \(f(n)\) это не 5, то положим \(x_n = 5\). Иначе положим \(x_n = 4\).
    \end{itemize}
    Так как \(f\) биекция, то найдется \(k \in \N: x = f(k)\). Так как \(x\) состоит только из 4 и 5, то у числа \(x\) единственная десятичная запись. Тогда и у числа \(f(k)\) тоже единственная десятичная запись, а значит записи \(x\) и \(f(k)\) совпадают. Но по построению \(k\)-ая цифра \(x\) не равна \(k\)-ой цифре \(f(k)\). Противоречие, а значит \(f\) не существует.
\end{proof}

\subsection{Теорема Кантора-Бернштейна}

\begin{lemma}
    \label{lm:cantor-second-th}
    Пусть \(h: A \rightarrow A\) --- инъекция. Пусть \(h(A) \subset E \subset A\). Тогда \(\card E = \card A\). 
\end{lemma}
\begin{proof}
    Не умаляя общности, считаем что \(A\) --- бесконечно, \(h(A) \subsetneq E \subsetneq A\).
    
    Заметим, что \(\card h(A) = \card A\) так как \(h\) биекция между \(A\) и \(h(A)\). Рассмотрим последовательность множеств 
    \[(A_0 = A, A_1 = h(A_0), \ldots, A_n = h(A_{n - 1}), \ldots)\]
    
    Покажем, что \(A_n \subset A_{n - 1}\) для любого \(n\) по индукции: база, \(n = 1\), выполнена по условию.  Пусть для всех \(1 \leqslant k \leqslant n\) выполнено что \(A_k \subset A_{k - 1}\). Тогда если \(h(A_n) \not\subset A_n\), то существует такая точка \(x\), что: 
    \[x \in A_n \land h(x) \not\in A_n\]
    
    Обозначим, \(s = \max\set{i \in \N_0 | x \in A_i}\). Очевидно \(n > s \geqslant 1\) (т.к. \(h(x)\) --- образ какого-то элемента из \(A_0\)).  По определению \(s\) имеем, что 
    \[y_0 = h(x) \in A_s \setminus A_{s + 1}\]
    
    Но при этом, \(x \in A_s\), а значит, по индукционному предположению, \(h(x) \in A_{s + 1}\) --- противоречие. 
    
    Теперь, обозначим:
    \[C = \bigcap\limits_{i = 0}^{\infty} A_i\]
    
    Заметим что в силу инъективности \(h\): 
    \[h(A_n \setminus A_{n + 1}) = h(A_n) \setminus h(A_{n + 1}) = A_{n + 1} \setminus A_{n + 2} \]
    
    Пусть \(P = A \setminus E,\ Q = E \setminus A_1\). Аналогично \(\set{A_n}_{n = 0}^{\infty}\) построим \(\set{P_n}_{n = 0}^{\infty}\) и \(\set{Q_n}_{n = 0}^{\infty}\). Ясно по индукции, что для любого \(n\): \(P_n \cup Q_n = A_n \setminus A_{n + 1}\). 
    
    Построим биекцию \(H: A \rightarrow E\):
    \[H(x) = \begin{cases}h(x), x \in P_n \\ x, x \in Q_n, \\ x, x \in C \end{cases}\]
\end{proof}
    
\begin{theorem}[Теорема Кантора-Бернштейна или вторая теорема Кантора]
    \label{th:second-cantor-th}
    Пусть \(A, B\) множества. Если \(\card A \leqslant \card B\) и \(\card B \leqslant \card A\) то \(\card A = \card B\) 
\end{theorem}
\begin{proof}
    Очевидным образом следует из леммы \ref{lm:cantor-second-th}:
    
    Пусть \(f: A \rightarrow B; g: B \rightarrow A\) --- инъекции. Тогда:
    \[g(f(A)) \subset g(B) \subset A \underset{\text{\tiny по лемме}}{\Longrightarrow} \card B = \card g(B) = \card A\]
\end{proof}

Иллюстрации: 
\begin{figure}[h!]
    \centering
    \begin{tikzpicture}
        \filldraw[fill=blue!5!white, draw=blue] (0, 0) circle(3);
        \draw[thick, dashed, red] (0, 0) circle(2.5);
        \filldraw[fill=blue!10!white, draw=blue] (0, 0) circle(2);
        \filldraw[fill=blue!15!white, draw=blue!10!white] (0, 0) circle(1.5);
        \draw[thick, dashed, blue]  (0, 0) circle(1.5);
        \filldraw[fill=blue!20!white, draw=blue] (0, 0) circle(1);
        \filldraw[fill=blue!25!white, draw=blue] (0, 0) circle(0.5);
        
        \node at (0, 0) (nodeC) {\(C\)};
        \node at (0, 2.3) (nodeE) {\(E\)};
        \node at (2.75, 0) (nodeA0) {\(A_0\)};
        \node at (1.75, 0) (nodeA1) {\(A_1\)};
        \node at (1.25, 0) (nodeA2) {\(\ldots\)};
        \node at (0.75, 0) (nodeA2) {\(A_n\)};
        
        \filldraw[fill=blue!5!white, draw=blue] (7, 0) circle(3);
        \filldraw[fill=red!10!white, draw=red,dashed] (7, 0) circle(2.5);
        
        \filldraw[fill=blue!10!white, draw=blue] (7, 0) circle(2);
        \filldraw[fill=red!15!white, draw=red,dashed] (7, 0) circle(1.75);
        
        \filldraw[fill=blue!15!white, draw=blue] (7, 0) circle(1.5);
        \filldraw[fill=red!20!white, draw=red,dashed] (7, 0) circle(1.25);
        
        \filldraw[fill=blue!20!white, draw=blue] (7, 0) circle(1);
        \filldraw[fill=red!25!white, draw=red,dashed] (7, 0) circle(0.75);
        
        \filldraw[fill=blue!25!white, draw=blue] (7, 0) circle(0.5);
        
        \node at (7, 0) (nodeC) {\(C\)};
        \node at (7, 2.25) (nodeQ) {\(Q_0\)};
        \node at (7, 2.7) (nodeP) {\(P_0\)};
        
    \end{tikzpicture}
    \caption{К лемме \ref{lm:cantor-second-th}}
    \label{fig:example-lm-cantor-second-th}
\end{figure}

\begin{figure}[h!]
 \centering
    \begin{tikzpicture}
        \filldraw[fill=blue!5!white, draw=blue] (0, 0) circle(3);
        \filldraw[fill=red!10!white, draw=red] (7, 0) circle(3);
        \filldraw[fill=red!15!white, draw=red] (7, 0) circle(2);
        \filldraw[fill=blue!10!white, draw=blue] (0, 0) circle(2);
        \filldraw[fill=blue!15!white, draw=blue] (0, 0) circle(1);
        \draw[thick, ->] (0, 2.75) -- (7, 1.7);
        \draw[thick, ->] (0, -2.75) -- (7, -1.7);
        \draw[thick, <-] (0, 0.7) -- (7, 1.5);
        \draw[thick, <-] (0, -0.7) -- (7, -1.5);

        \node at (0, 1.5) (nodeC) {\(g(B)\)};
        \node at (0, 2.5) (nodeE) {\(A\)};
        \node at (0, 0) (nodeA0) {\(g(f(A))\)};
        \node at (7, 2.5) (nodeA1) {\(B\)};
        \node at (7, 0) (nodeA2) {\(f(A)\)};
    \end{tikzpicture}
    \caption{К теореме \ref{th:second-cantor-th}}
    \label{fig:example-th-cantor-second-th}
\end{figure}

\subsection{Теорема Кантора о неравномощности множества и его булеана}

\begin{theorem}[Третья теорема Кантора]
    \label{th:third-cantor-th}
    \[\card A \neq \card 2^A\]
\end{theorem}
\begin{proof}
    Предположим обратное. Тогда есть биекция: \(f: A \rightarrow 2^A\).
    
    Пусть \(\mathcal A \subset 2^A\) --- множество всех таких \(x\) что \(x \not\in f(x)\). Но тогда рассмотрим прообраз \(f^{-1}(\mathcal A) = x_0\): 
    \begin{align*}
        x_0 \in \mathcal A \Rightarrow x_0 \not\in \mathcal A \\
        x_0 \not\in \mathcal A \Rightarrow x_0 \in \mathcal A
    \end{align*}
    Получаем противоречие. 
\end{proof}

\section{Иерархия бесконечностей}

\subsection{Счетные множества}
\begin{theorem}
    Подмножество счетного множества не более чем счетно
\end{theorem}
\begin{theorem}
    Конечное декартово произведение счетных множеств счетно
\end{theorem}
\begin{theorem}
    Счетное объединение не более чем счетных множеств счетно
\end{theorem}

Пример использования аксиомы выбора

\subsection{Несчетные множества и континуум}

О разнице между несчетностью и континуумом. 

Мощности строго больше континуума. Иерархия алефов

О промежуточных мощностях. Гипотеза континуума. 

\end{document}
