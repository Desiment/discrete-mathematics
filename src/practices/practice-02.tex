%
\documentclass[a4paper,12pt,twoside]{article}
\usepackage{iftex}

%% Разрешить компиляцию только с движком LuaTex
\ifLuaTeX
\else
    \newlinechar 64\relax
    \errorcontextlines -1\relax
    \immediate\write20{@
        ************************************************@
        * LuaLaTex is required to compile this document.@
        * Sorry!@
        ************************************************}%
    \batchmode\read -1 to \@tempa
\fi

%% Для русификации достаточно подключить пакет fontspec и
%% выбрать Unicode шрифт в котором есть кириллические глифы. Ниже
%% основным шрифтом выбирается Unicode версия шрифта Computer Modern с заcечками
\usepackage{fontspec}
\setmainfont{CMU Serif}
\setsansfont{CMU Sans Serif}
\setmonofont{CMU Typewriter Text}

%% В XeLaTex или LuaLaTeX альтернативой известного пакета babel является пакет polyglossia.
%% Теперь у нас будут переносы слов
\usepackage{polyglossia}
\setdefaultlanguage{russian}
\setotherlanguage{english}

\usepackage[autostyle]{csquotes} % Правильные кавычки в зависимости от языка
\usepackage{totcount}
\usepackage{setspace}

% ToDo:
% [] MathNote   : \sum[] syntax
% [] MathNote   : \tikzmatrix 
% [] Layout     : A4 geometry
% [] Layout     : Define colors
% [] References : Load knowledge
% [] References : Create simple clever refs
% [] Fix hidding enviroment

%Отключить предупреждения об кастномной использовании пакетов "You have requested package..."
\usepackage{silence}
\WarningFilter{latex}{You have requested package}

\usepackage{xparse}
\usepackage{configuration/floats}
\usepackage{configuration/layout}
\usepackage{configuration/references}
\usepackage{configuration/mathnote}

\sloppy
% Окружения для набора задач
\newcounter{boxlblcounter}  

\newcommand{\task}[1]{\fbox{\begin{minipage}{4em}\centering\it #1\end{minipage}}}
\newcommand{\makeboxlabel}[1]{\fbox{\begin{minipage}{2em}\centering\it #1\end{minipage}}\hfill}% \hfill fills the label box
\newenvironment{tasklist}
  {\begin{list}
    {\arabic{boxlblcounter}}
    {\usecounter{boxlblcounter}
     \setlength{\labelwidth}{3em}
     \setlength{\labelsep}{0em}
     \setlength{\itemsep}{2pt}
     \setlength{\leftmargin}{1.5cm}
     \setlength{\rightmargin}{2cm}
     \setlength{\itemindent}{0em} 
     \let\makelabel=\makeboxlabel
    }
  }{\end{list}}


\newboolean{ShowHint}
\newboolean{ShowSolution}

% Подсказка
\newcommand{\hint}[1]{\ifthenelse{\boolean{ShowHint}}{\noindent\rotatebox[origin=c]{180}{\noindent
\begin{minipage}[t]{\linewidth} \noindent \it Указание: #1 \end{minipage}}}{}}

% solution: окружение для набора решений
% solution*: форсировано показывает решение, независимо от флага
\makeatletter
\newenvironment{solution*}[1][\text{Решение}]{ 
  \par
  \pushQED{\qed}%
  \normalfont
  \topsep0pt \partopsep3pt
  \trivlist
  \item[\hskip\labelsep
    \itshape
    #1\@addpunct{.}]\ignorespaces
}{
  \ifthenelse{\boolean{ShowSolution}}{
    \popQED\endtrivlist\@endpefalse
    \addvspace{6pt plus 6pt} % some space after
  }{\end{hidden}}
}
\makeatother

%https://tex.stackexchange.com/questions/533218/hiding-an-environment-that-contains-minted-code
%https://tex.stackexchange.com/questions/38150/in-lualatex-how-do-i-pass-the-content-of-an-environment-to-lua-verbatim
\RequirePackage{luacode}
\begin{luacode*}
do 
    function eat_buffer(buf)
        i,j = string.find(buf,"\\end{solution}")
        if i==nil then return "" else return nil end
    end
    function start_proccesing_solution(eat)
        if eat then luatexbase.add_to_callback('process_input_buffer', eat_buffer, 'eat_buffer') end
    end
    function stop_proccesing_solution(eat)
        if eat then luatexbase.remove_from_callback('process_input_buffer', 'eat_buffer') end
    end
end
\end{luacode*}
%https://tex.stackexchange.com/questions/537219/conditionals-inside-newcommand-with-empty-argument
%https://tex.stackexchange.com/questions/63223/xparse-empty-arguments
\ExplSyntaxOn
\DeclareExpandableDocumentCommand{\IfNoValueOrEmptyTF}{mmm}{\IfNoValueTF{#1}{#2}{\tl_if_empty:nTF {#1} {#2} {#3}}}
\NewDocumentCommand{\DefaultName}{mm}{\IfNoValueOrEmptyTF{#1}{#2}{#1}}
\ExplSyntaxOff

% WARNING!
% Buggy: one have always have
% \begin{solution}{NAME}
%   ....
% \end{solution}
% EVEN if NAME is empty
% if one starts some text on the same line as \begin{solution} or \end{solution} it could'not be hidden
\newenvironment{solution}[1]%
{
  \ifthenelse{\boolean{ShowSolution}}
    {\begin{solution*}[\DefaultName{#1}{\textit{Решение}}]\directlua{flag_eat=false}}
    {\directlua{flag_eat=true}}
  \directlua{start_proccesing_solution(flag_eat)}
}
{
  \ifthenelse{\boolean{ShowSolution}}{\end{solution*}}{}
  \directlua{stop_proccesing_solution(flag_eat)}
}


\newcommand{\PracticeSubject}{}
\newcommand{\PracticeName}{}
\newcommand{\PracticeGroup}{}
\newcommand{\PracticeCourse}{}
\newcommand{\PracticeDate}{}


\AtBeginDocument{%
    % Метаданные:
    \title{\PracticeName}%
    \date{\PracticeDate}%
    % Настраиваем колонтитулы
    \pagestyle{fancy}%
    \lhead{\PracticeGroup, курс \PracticeCourse}%
    \chead{\PracticeSubject. Практика}%
    \rhead{\PracticeDate}%

    % Название практики
    \section*{\strtitle}
}
\mmzset{memo dir = images/cache/\jobname}


\setboolean{ShowHint}{true}
\setboolean{ShowSolution}{false}

\renewcommand{\PracticeSubject}{Дискретная математика}
\renewcommand{\PracticeName}{Функции, множества и счётность}
\renewcommand{\PracticeGroup}{ПАДИИ}
\renewcommand{\PracticeCourse}{1}
\renewcommand{\PracticeDate}{31 января 2025}


\begin{document}
\begin{?}
    Пусть множество $A$ конечно, а множество $B$ счетно. Докажите, что тогда множество функций $f\colon A \to B$ счетно.
\end{?}
\ifthenelse{\boolean{ShowSolution}}{\vspace{-24pt}}{}
\begin{solution}{}
    Это множество есть \(\overbrace{B \times B \times B \ldots B}^{\text{|A| раз}}\). Известно, что конечное произведение счетных множеств само счетно.
\end{solution}
\begin{?}
    Функция называется периодической, если для некоторого $T \neq 0$ и любого $x$ выполняется равенство $f(x) = f(x + T) = f(x - T)$. Докажите, что множество периодических функций $f \colon \Z \to \Z$  счетно.
\end{?}
\begin{solution}{}
    Чтобы задать периодическую функция \(\Z \to \Z\) фиксированно периода \(T\), достаточно указать её первые \(T\) значений. Поэтому для фиксированного \(T\), множество \(T\)-периодических функций \(\Z \to \Z\) счетно в силу предыдущей задачи. Так как периодов всего счетное число, то множество периодических функций есть счетное объединение счетных множеств и потому является счетным множеством.
\end{solution}
\begin{?}
    Докажите, что множество точек строгого локального максимума любой функции вещественного аргумента конечно или счётно.
\end{?}
\begin{solution}{}
    Для заметим что, если \(x\) --- строгий локальный максимум \(f\) в окрестности \(V\), \(x' \in V\) и строгий локальный максимум \(f\) в окрестности \(U\), причем \(f(x) > f(x')\), то, либо \(U \prec x\), либо \(x \prec U\) (т.е. все \(U\) целиком либо справа, либо слева от \(x\)) (так как \(x \not\in U\)). В частности, если взять в окрестности локального максимума две рациональные точки, одну слева от него, другую справа от него, то разным локальным максимумам будут отвечать разные пары рациональных точек. Что и требовалось доказать.
\end{solution}
\begin{?}
    Докажите, что множество точек разрыва неубывающей функции вещественного аргумента конечно или счётно.
\end{?}
\begin{solution}{}
    Действительно, раз функция неубывающая, то точки её разрыва находятся во взаимно-однозначном соответствии с интервалами вида \([f(x-); f(x+)]\), где \(x\) --- точка разрыва. Но в каждом таком интервале обязательно содержится рациональная точка, и разным интервалам будут отвечать разные рациональные точки, поэтому их не более чем счётное число.
\end{solution}
\begin{?}
    Говорят, что $g \colon B \to A$ является левой обратной (соответственно правой обратной) к $f$, если $g \circ f = \mathrm{id}_{A}$ (соответственно $f \circ g = \mathrm{id}_{B}$).
    \begin{enumerate}[noitemsep, topsep=0pt, parsep=0pt]
        \item Приведите примеры, когда левая обратная не является правой обратной и наоборот.
        \item Может ли такое случиться для конечных множеств?
        \item Может ли быть так, что у одной функции есть и левая и правая обратные, но они различны?
        \item Для каких функций существует левая обратная?
        \item Для каких функций существует правая обратная? 
    \end{enumerate}
\end{?}
\begin{solution}{}
    См{.}~заметку об отображениях. 
\end{solution}
\end{document}

