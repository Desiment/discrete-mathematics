%
\documentclass[a4paper,12pt,twoside]{article}
\usepackage{iftex}

%% Разрешить компиляцию только с движком LuaTex
\ifLuaTeX
\else
    \newlinechar 64\relax
    \errorcontextlines -1\relax
    \immediate\write20{@
        ************************************************@
        * LuaLaTex is required to compile this document.@
        * Sorry!@
        ************************************************}%
    \batchmode\read -1 to \@tempa
\fi

%% Для русификации достаточно подключить пакет fontspec и
%% выбрать Unicode шрифт в котором есть кириллические глифы. Ниже
%% основным шрифтом выбирается Unicode версия шрифта Computer Modern с заcечками
\usepackage{fontspec}
\setmainfont{CMU Serif}
\setsansfont{CMU Sans Serif}
\setmonofont{CMU Typewriter Text}

%% В XeLaTex или LuaLaTeX альтернативой известного пакета babel является пакет polyglossia.
%% Теперь у нас будут переносы слов
\usepackage{polyglossia}
\setdefaultlanguage{russian}
\setotherlanguage{english}

\usepackage[autostyle]{csquotes} % Правильные кавычки в зависимости от языка
\usepackage{totcount}
\usepackage{setspace}

% ToDo:
% [] MathNote   : \sum[] syntax
% [] MathNote   : \tikzmatrix 
% [] Layout     : A4 geometry
% [] Layout     : Define colors
% [] References : Load knowledge
% [] References : Create simple clever refs
% [] Fix hidding enviroment

%Отключить предупреждения об кастномной использовании пакетов "You have requested package..."
\usepackage{silence}
\WarningFilter{latex}{You have requested package}

\usepackage{xparse}
\usepackage{configuration/floats}
\usepackage{configuration/layout}
\usepackage{configuration/references}
\usepackage{configuration/mathnote}

\sloppy
% Окружения для набора задач
\newcounter{boxlblcounter}  

\newcommand{\task}[1]{\fbox{\begin{minipage}{4em}\centering\it #1\end{minipage}}}
\newcommand{\makeboxlabel}[1]{\fbox{\begin{minipage}{2em}\centering\it #1\end{minipage}}\hfill}% \hfill fills the label box
\newenvironment{tasklist}
  {\begin{list}
    {\arabic{boxlblcounter}}
    {\usecounter{boxlblcounter}
     \setlength{\labelwidth}{3em}
     \setlength{\labelsep}{0em}
     \setlength{\itemsep}{2pt}
     \setlength{\leftmargin}{1.5cm}
     \setlength{\rightmargin}{2cm}
     \setlength{\itemindent}{0em} 
     \let\makelabel=\makeboxlabel
    }
  }{\end{list}}


\newboolean{ShowHint}
\newboolean{ShowSolution}

% Подсказка
\newcommand{\hint}[1]{\ifthenelse{\boolean{ShowHint}}{\noindent\rotatebox[origin=c]{180}{\noindent
\begin{minipage}[t]{\linewidth} \noindent \it Указание: #1 \end{minipage}}}{}}

% solution: окружение для набора решений
% solution*: форсировано показывает решение, независимо от флага
\makeatletter
\newenvironment{solution*}[1][\text{Решение}]{ 
  \par
  \pushQED{\qed}%
  \normalfont
  \topsep0pt \partopsep3pt
  \trivlist
  \item[\hskip\labelsep
    \itshape
    #1\@addpunct{.}]\ignorespaces
}{
  \ifthenelse{\boolean{ShowSolution}}{
    \popQED\endtrivlist\@endpefalse
    \addvspace{6pt plus 6pt} % some space after
  }{\end{hidden}}
}
\makeatother

%https://tex.stackexchange.com/questions/533218/hiding-an-environment-that-contains-minted-code
%https://tex.stackexchange.com/questions/38150/in-lualatex-how-do-i-pass-the-content-of-an-environment-to-lua-verbatim
\RequirePackage{luacode}
\begin{luacode*}
do 
    function eat_buffer(buf)
        i,j = string.find(buf,"\\end{solution}")
        if i==nil then return "" else return nil end
    end
    function start_proccesing_solution(eat)
        if eat then luatexbase.add_to_callback('process_input_buffer', eat_buffer, 'eat_buffer') end
    end
    function stop_proccesing_solution(eat)
        if eat then luatexbase.remove_from_callback('process_input_buffer', 'eat_buffer') end
    end
end
\end{luacode*}
%https://tex.stackexchange.com/questions/537219/conditionals-inside-newcommand-with-empty-argument
%https://tex.stackexchange.com/questions/63223/xparse-empty-arguments
\ExplSyntaxOn
\DeclareExpandableDocumentCommand{\IfNoValueOrEmptyTF}{mmm}{\IfNoValueTF{#1}{#2}{\tl_if_empty:nTF {#1} {#2} {#3}}}
\NewDocumentCommand{\DefaultName}{mm}{\IfNoValueOrEmptyTF{#1}{#2}{#1}}
\ExplSyntaxOff

% WARNING!
% Buggy: one have always have
% \begin{solution}{NAME}
%   ....
% \end{solution}
% EVEN if NAME is empty
% if one starts some text on the same line as \begin{solution} or \end{solution} it could'not be hidden
\newenvironment{solution}[1]%
{
  \ifthenelse{\boolean{ShowSolution}}
    {\begin{solution*}[\DefaultName{#1}{\textit{Решение}}]\directlua{flag_eat=false}}
    {\directlua{flag_eat=true}}
  \directlua{start_proccesing_solution(flag_eat)}
}
{
  \ifthenelse{\boolean{ShowSolution}}{\end{solution*}}{}
  \directlua{stop_proccesing_solution(flag_eat)}
}


\newcommand{\PracticeSubject}{}
\newcommand{\PracticeName}{}
\newcommand{\PracticeGroup}{}
\newcommand{\PracticeCourse}{}
\newcommand{\PracticeDate}{}


\AtBeginDocument{%
    % Метаданные:
    \title{\PracticeName}%
    \date{\PracticeDate}%
    % Настраиваем колонтитулы
    \pagestyle{fancy}%
    \lhead{\PracticeGroup, курс \PracticeCourse}%
    \chead{\PracticeSubject. Практика}%
    \rhead{\PracticeDate}%

    % Название практики
    \section*{\strtitle}
}
\mmzset{memo dir = images/cache/\jobname}


\setboolean{ShowHint}{false}
\setboolean{ShowSolution}{true}

\renewcommand{\PracticeSubject}{Дискретная математика}
\renewcommand{\PracticeName}{Остовные деревья}
\renewcommand{\PracticeGroup}{ПМИ}
\renewcommand{\PracticeCourse}{2}
\renewcommand{\PracticeDate}{27 янвая 2025}

\begin{document}
\begin{?}
    Рассмотрим 3-регулярный граф $G_m$, представляющий собой ожерелье из $m$ графов \enquote{воздушный змей} (на рис.~1 приведен пример для $m = 6$). Подсчитайте количество остовных деревьев в таком графе для произвольного значения параметра $m>1$.
    \begin{figure}[h]
        \centering
        \includetikz{practice-02-img-01.tikz}
        \caption{}
    \end{figure}
\end{?}
\begin{solution}{}
    Поймем как устроены остовные деревья в таком графе. Чтобы получить остовное дерево, надо, условно говоря, разомкнуть циклы во всех воздушных змеях и один \enquote{общий цикл}. Это можно сделать двумя способами. 
    \begin{enumerate}[noitemsep,topsep=0pt,parsep=0pt]
        \item Разорвать одно из связывающих \enquote{воздушных змеев} ребер и в каждом воздушном змее выбрать остовное дерево.
        \item Выбрать один из воздушных змеев, разомкнуть по нему и во всех оставшихся построить остовное дерево (в случае если $m > 1$).
    \end{enumerate}
    Схематично это изображено на рисунке ниже.
    \begin{center}
        \includetikz{practice-02-solution-img-01.tikz}
    \end{center}

    Начнем с того что посчитаем, сколькими способами можно выбрать остовное дерево в каждом воздушном змее. Матрица Лапласа для графа воздушного змея $G_1$ есть:
    \[
        L(G_1) = \begin{pmatrix}
            2  & -1 &  0 & -1 \\
            -1 &  3 & -1 & -1 \\
            0  & -1 &  2 & -1 \\
            -1 & -1 & -1 &  3
        \end{pmatrix}
    \]
    По матричной теореме \(\mathrm{st}(G_1)\) есть:
    \[
        \mathrm{st}(G_1) = \abs{\det(L_{4, 4}(G_1))} = \left|\begin{vmatrix}
            2  & -1 &  0 \\
            -1 &  3 & -1 \\
            0  & -1 &  2\\
        \end{vmatrix}\right| = 8
    \]
    Значит, для $G_m$ есть $m \cdot 8^m$ остовных деревьев первого вида. Для того чтобы найти количество остовных деревьев второго вида, необходимо подсчитать сколько есть способов в $G_1$ составить остовный лес из двух компонент разделяющий вершины степени 2. 
    \newpage 
    \noindent Схематично эти способы изображены на рисунке ниже
    \begin{center}
        \includetikz{practice-02-solution-img-02.tikz}
    \end{center}
    Итого остовных деревьев второго вида \(8 \cdot m \cdot 8^{m-1} = m \cdot 8^m\) и значит \(\mathrm{st}(G_m) = 2m \cdot 8^n\) (если $m > 1$).

    \noindent \textbf{Ответ}: \(\mathrm{st}(G_m) = 2m \cdot 8^n\)
\end{solution}

\begin{?} Подсчитайте
    \begin{itemize}[noitemsep, topsep=0pt, parsep=0pt]
        \item количество остовных деревьев графа $K_{2,3}$.
        \item количество остовных деревьев графа $K_{2,n}$.
        \item количество остовных деревьев графа $K_{m,n}$.
    \end{itemize}
\end{?}
\begin{solution}{}
Для начала комбинаторно посчитаем число остовных деревьев графа $K_{2,n}$. 
\begin{center}
    \includetikz{practice-02-solution-img-03.tikz}
\end{center}
Заметим что в остовном дереве всегда будет путь $a \to b$, и так как граф двудольный этот путь всегда будет иметь вид $a \rightsquigarrow i \rightsquigarrow b$, $i \in \set{1, \ldots, n}$. Значит каждое остовное дерево имеет следующий вид. Между вершинами $a$ и $b$ находится одна смежная вершина $i$, а все вершины из $\set{1, \ldots, n} \setminus \set{i}$ являются либо листами $a$, либо листами $b$. Итого: $n$ способов выбрать смежную вершину, и $2^{n-1}$ способ развесить оставшиеся как листы.

\begin{center}
    \includetikz{practice-02-solution-img-04.tikz}
\end{center}

\noindent \textbf{Ответ}: \(\mathrm{st}(K_{2, n}) = n \cdot 2^{n -1}\).


Для подсчета $\mathrm{st}(K_{n, m})$ воспользуемся матричной теоремой. Матрица Лапласа графа $K_{n,m}$ это блочная матрица следующего вида. 
\[
    L(K_{n, m})=  \begin{pmatrix}
        m I_{n} & \mathds{1}_{n \times m} \\
        \mathds{1}_{m \times n} & n I_{m}
    \end{pmatrix},
\]
где $\mathds{1}_{n \times m}$ матрица $n \times m$ с единицей во всех ячейках. Определитель её минора можно расписать как определитель блочной матрицы:
\[
    \det(L_{1, 1}) = \begin{vmatrix}
        m I_{n-1} & \mathds{1}_{(n-1) \times m} \\
        \mathds{1}_{m \times (n-1)} & n I_{m}
    \end{vmatrix} = \det\mleft( m I_{n-1} \mright) \cdot \det\mleft(n I_{m} - \mathds{1}_{m \times (n-1)} \cdot (m I_{n-1})^{-1} \cdot \mathds{1}_{(n-1) \times m}\mright)
\]
понятно что $\det\mleft( m I_{n-1} \mright) = m^{n-1}$, найдем второй множитель. Для этого запишем:
\[
    M_{n, m} = \mathds{1}_{m \times (n-1)} \cdot (m I_{n-1})^{-1} \cdot \mathds{1}_{(n-1) \times m} = \mathds{1}_{m \times (n-1)} \cdot \mleft(\frac{1}{m} I_{n-1}\mright) \cdot \mathds{1}_{(n-1) \times m} = \frac{1}{m} \mathds{1}_{m \times (n-1)}  \cdot \mathds{1}_{(n-1) \times m} = \frac{n-1}{m} \cdot \mathds{1}_{m \times m}
\]
Тогда
\begin{multline*}
    M_{n,m} = n I_{m} - \mathds{1}_{m \times (n-1)} \cdot (m I_{n-1})^{-1} \cdot \mathds{1}_{(n-1) \times m} = \frac{1}{m}\mleft(nm I_{m} - (n-1)\mathds{1}_{m \times m} \mright) =\\= \frac{1}{n}
    \begin{pmatrix}
        nm - (n - 1) & -(n - 1)     & \vdots  & -(n - 1) \\
        -(n - 1)     & nm - (n - 1) & \vdots & -(n - 1) \\
         \cdots      & \cdots       & \ddots & \vdots \\
        -(n - 1)     &-(n - 1)      & \cdots &  nm - (n - 1) \\
    \end{pmatrix}
\end{multline*}
Начнем преобразовать эту матрицу. Для начала вычтем последний столбец из всех остальных:
\[
    M_{n, m} \sim \frac{1}{m}\begin{pmatrix}
    nm    & 0     & \vdots & 0      & -(n-1)\\
    0     & nm    & \vdots & 0      & -(n-1)\\
   \cdots &\cdots & \ddots & \vdots & \vdots \\
    0     & 0     & \cdots &  nm    & -(n-1)\\
    -nm   & -nm   & \cdot  & -nm    & nm -(n-1) 
\end{pmatrix}
\]
Теперь прибавим к последнему ряду все предыдущие:
\[
    M_{n, m} \sim \frac{1}{m}\begin{pmatrix}
    nm    & 0     & \vdots & 0      & -(n-1)\\
    0     & nm    & \vdots & 0      & -(n-1)\\
   \cdots &\cdots & \ddots & \vdots & \vdots \\
    0     & 0     & \cdots &  nm    & -(n-1)\\
    0     & 0     & \cdot  & 0      & nm - m(n-1)
\end{pmatrix}
\]
Имеем \(nm - m(n-1) = m\), получили что:
\[
    M_{n, m} \sim \begin{pmatrix}
    n    & 0     & \vdots & 0      & -\frac{1}{m}(n-1)\\
    0     & n    & \vdots & 0      & -\frac{1}{m}(n-1)\\
   \cdots &\cdots & \ddots & \vdots & \vdots \\
    0     & 0     & \cdots &  n    & -\frac{1}{m}(n-1)\\
    0     & 0     & \cdot  & 0      & 1
\end{pmatrix}
\]
В частности \(\det M_{n, m} = n^{m - 1}\) и тогда:
\[
    L_{1, 1} = \det\mleft( m I_{n-1} \mright) \cdot \det M_{n, m} = m^{n-1}n^{m-1}
\]

\noindent \textbf{Ответ}: \(\mathrm{st}(K_{m, n}) =  m^{n-1}n^{m-1}\).
\end{solution}

\begin{?}
    Подсчитайте количество остовных деревьев графа $G_n$ «веер» (см.~рис.~2). Можно ли выразить это количество через числа Фибоначчи?
    \begin{figure}[H]
        \centering
        \includetikz{practice-02-img-02.tikz}
    \end{figure}
\end{?}
\begin{solution}{}
    Воспользуемся рекуррентным соотношением: $\mathrm{st}(G) = \mathrm{st}(G \ast e) + \mathrm{st}(G - e)$ для любого ребра $e$ (стягивание должно давать при этом граф с кратными ребрами). Будем рассматривать ребро $(0; n)$. Получившийся после его стягивания граф назовем $L_{n-1}$.  Тогда можем схематично изобразить вывод рекуррентного соотношения:

    \vspace{12pt}

    \begin{center}
        \includetikz{practice-02-solution-img-05.tikz}
    \end{center}

    \vspace{12pt}

    \noindent Получили что:
    \[
        \begin{cases}
            \mathrm{st}(G_n) = \mathrm{st}(G_{n-1}) + \mathrm{st}(L_{n-1}) \\
            \mathrm{st}(L_n) = \mathrm{st}(G_{n}) + \mathrm{st}(L_{n-1}) \\
        \end{cases}
    \]
    Сделаем небольшие алгебраические преобразования:
    \[
        \begin{cases}
            \mathrm{st}(G_n) - \mathrm{st}(G_{n-1}) = \mathrm{st}(L_{n-1}) \\
            \mathrm{st}(L_n) - \mathrm{st}(G_{n}) = \mathrm{st}(L_{n-1}) \\
        \end{cases}
    \]
    Откуда \(\mathrm{st}(L_n) = 2 \mathrm{st}(G_n) - \mathrm{st}(G_{n-1})\), и значит
    \[
        \mathrm{st}(G_n) = 3 \cdot \mathrm{st}(G_{n-1}) - \mathrm{st}(G_{n-2})
    \] 
    Мы также имеем что $\mathrm{st}(G_1) = 1, \mathrm{st}(L_1) = 2, \mathrm{st}(G_2) = 3,  \mathrm{st}(L_1) = 5$. Честно решая рекуррентное соотношение или методом пристального взгляда можно заметить что $\mathrm{st}(G_n) = F_{2n - 1}, \mathrm{st}(L_n) = F_{2n}$, где $F_{k}$ --- $k$-ое число Фибоначчи.
    
    \noindent\textbf{Ответ} $\mathrm{st}(G_n) = F_{2n - 1}$.
\end{solution}

\begin{?}
    Используя матричную теорему о деревьях, подсчитайте количество всех деревьев на $n$ вершинах.
\end{?}
\begin{solution}{}
    Запишем матрицу Лапласа для $K_n$:
    \[
        L(K_{n}) = \begin{pmatrix}
            n - 1  &   - 1  &   - 1  & \ldots &  - 1   \\
              - 1  & n - 1  &   - 1  & \vdots &  - 1   \\
            \cdots & \cdots & \ddots & \vdots & \vdots \\
              -1   & \cdots & \cdots &  n - 1 &   -1   \\
              -1   & \cdots & \cdots &    - 1 &  n - 1 \\
        \end{pmatrix}
    \]
    Необходимо найти определитель её минора. Возьмем $L_{1, 1}$, понятно что:
    \[
        \det(L_{1, 1}) = \det\mleft(nI_{n - 1} - \mathds{1}_{(n - 1) \times (n - 1)}\mright)
    \]
    Посмотрим на матрицу $\mathds{1}_{(n - 1) \times (n - 1)}$. Её ранг равен 1, и при этом:
    \[
        \mathds{1}_{(n - 1) \times (n - 1)} \cdot \begin{pmatrix}
            1 \\ \vdots \\ 1
        \end{pmatrix} = \begin{pmatrix}
            n - 1 \\ \vdots \\ n - 1
        \end{pmatrix} 
    \]
    поэтому её собственные числа это $(n - 1)$ кратности 1 и $0$ кратности $n - 2$. В частности, эта матрица может быть диагонализирована сопряжением на какую-то матрицу $P$:
    \[
        P^{-1} \cdot \mathds{1}_{(n - 1) \times (n - 1)} \cdot P = \begin{pmatrix}
            n - 1  & 0      & \cdots & 0 \\
              0    & 0      & \vdots & \vdots \\
            \vdots & \cdots & \ddots &  0  \\
              0    & \cdots &   0    &  0
        \end{pmatrix} = \mathrm{diag}(n-1, 0, \ldots, 0)
    \]
    Но тогда, по свойствам определителя
    \begin{multline*}
        \det(L_{1, 1}) = \det(P^{-1} \cdot L_{1, 1} \cdot P) =  \det\big(\underbrace{P^{-1} \cdot nI_{n - 1} \cdot P}_{ nI_{n - 1} } - \underbrace{P^{-1} \cdot \mathds{1}_{(n - 1) \times (n - 1)} \cdot P}_{\mathrm{diag}(n-1, 0, \ldots, 0)}\big) = \\ \det(\mathrm{diag}(1, n, \ldots, n)) = n^{n-2}
    \end{multline*}
\end{solution}
\begin{?}
    Подсчитайте количество различных непомеченных остовных деревьев графа $K_{2,n}$.
\end{?}
\begin{solution}{}
    Структура остовных деревьев для такого графа была описана во второй задаче. Так как вершины теперь непочемены, чтобы задать дерево надо указать сколько листьев у одной вершины и другой вершины из доли с двумя вершинами. Иначе говоря, нас интересует количество неупорядоченных пар $x, y$ таких что $x + y = n - 1$ ($x$ и $y$ --- количество листьев). Таких пар: $\ceil{\frac{n}{2}}$. 

    \noindent\textbf{Ответ} $\ceil{\frac{n}{2}}$.
\end{solution}
\end{document}

