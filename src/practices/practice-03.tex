%
\documentclass[a4paper,12pt,twoside]{article}
\usepackage{iftex}

%% Разрешить компиляцию только с движком LuaTex
\ifLuaTeX
\else
    \newlinechar 64\relax
    \errorcontextlines -1\relax
    \immediate\write20{@
        ************************************************@
        * LuaLaTex is required to compile this document.@
        * Sorry!@
        ************************************************}%
    \batchmode\read -1 to \@tempa
\fi

%% Для русификации достаточно подключить пакет fontspec и
%% выбрать Unicode шрифт в котором есть кириллические глифы. Ниже
%% основным шрифтом выбирается Unicode версия шрифта Computer Modern с заcечками
\usepackage{fontspec}
\setmainfont{CMU Serif}
\setsansfont{CMU Sans Serif}
\setmonofont{CMU Typewriter Text}

%% В XeLaTex или LuaLaTeX альтернативой известного пакета babel является пакет polyglossia.
%% Теперь у нас будут переносы слов
\usepackage{polyglossia}
\setdefaultlanguage{russian}
\setotherlanguage{english}

\usepackage[autostyle]{csquotes} % Правильные кавычки в зависимости от языка
\usepackage{totcount}
\usepackage{setspace}

% ToDo:
% [] MathNote   : \sum[] syntax
% [] MathNote   : \tikzmatrix 
% [] Layout     : A4 geometry
% [] Layout     : Define colors
% [] References : Load knowledge
% [] References : Create simple clever refs
% [] Fix hidding enviroment

%Отключить предупреждения об кастномной использовании пакетов "You have requested package..."
\usepackage{silence}
\WarningFilter{latex}{You have requested package}

\usepackage{xparse}
\usepackage{configuration/floats}
\usepackage{configuration/layout}
\usepackage{configuration/references}
\usepackage{configuration/mathnote}

\sloppy
% Окружения для набора задач
\newcounter{boxlblcounter}  

\newcommand{\task}[1]{\fbox{\begin{minipage}{4em}\centering\it #1\end{minipage}}}
\newcommand{\makeboxlabel}[1]{\fbox{\begin{minipage}{2em}\centering\it #1\end{minipage}}\hfill}% \hfill fills the label box
\newenvironment{tasklist}
  {\begin{list}
    {\arabic{boxlblcounter}}
    {\usecounter{boxlblcounter}
     \setlength{\labelwidth}{3em}
     \setlength{\labelsep}{0em}
     \setlength{\itemsep}{2pt}
     \setlength{\leftmargin}{1.5cm}
     \setlength{\rightmargin}{2cm}
     \setlength{\itemindent}{0em} 
     \let\makelabel=\makeboxlabel
    }
  }{\end{list}}


\newboolean{ShowHint}
\newboolean{ShowSolution}

% Подсказка
\newcommand{\hint}[1]{\ifthenelse{\boolean{ShowHint}}{\noindent\rotatebox[origin=c]{180}{\noindent
\begin{minipage}[t]{\linewidth} \noindent \it Указание: #1 \end{minipage}}}{}}

% solution: окружение для набора решений
% solution*: форсировано показывает решение, независимо от флага
\makeatletter
\newenvironment{solution*}[1][\text{Решение}]{ 
  \par
  \pushQED{\qed}%
  \normalfont
  \topsep0pt \partopsep3pt
  \trivlist
  \item[\hskip\labelsep
    \itshape
    #1\@addpunct{.}]\ignorespaces
}{
  \ifthenelse{\boolean{ShowSolution}}{
    \popQED\endtrivlist\@endpefalse
    \addvspace{6pt plus 6pt} % some space after
  }{\end{hidden}}
}
\makeatother

%https://tex.stackexchange.com/questions/533218/hiding-an-environment-that-contains-minted-code
%https://tex.stackexchange.com/questions/38150/in-lualatex-how-do-i-pass-the-content-of-an-environment-to-lua-verbatim
\RequirePackage{luacode}
\begin{luacode*}
do 
    function eat_buffer(buf)
        i,j = string.find(buf,"\\end{solution}")
        if i==nil then return "" else return nil end
    end
    function start_proccesing_solution(eat)
        if eat then luatexbase.add_to_callback('process_input_buffer', eat_buffer, 'eat_buffer') end
    end
    function stop_proccesing_solution(eat)
        if eat then luatexbase.remove_from_callback('process_input_buffer', 'eat_buffer') end
    end
end
\end{luacode*}
%https://tex.stackexchange.com/questions/537219/conditionals-inside-newcommand-with-empty-argument
%https://tex.stackexchange.com/questions/63223/xparse-empty-arguments
\ExplSyntaxOn
\DeclareExpandableDocumentCommand{\IfNoValueOrEmptyTF}{mmm}{\IfNoValueTF{#1}{#2}{\tl_if_empty:nTF {#1} {#2} {#3}}}
\NewDocumentCommand{\DefaultName}{mm}{\IfNoValueOrEmptyTF{#1}{#2}{#1}}
\ExplSyntaxOff

% WARNING!
% Buggy: one have always have
% \begin{solution}{NAME}
%   ....
% \end{solution}
% EVEN if NAME is empty
% if one starts some text on the same line as \begin{solution} or \end{solution} it could'not be hidden
\newenvironment{solution}[1]%
{
  \ifthenelse{\boolean{ShowSolution}}
    {\begin{solution*}[\DefaultName{#1}{\textit{Решение}}]\directlua{flag_eat=false}}
    {\directlua{flag_eat=true}}
  \directlua{start_proccesing_solution(flag_eat)}
}
{
  \ifthenelse{\boolean{ShowSolution}}{\end{solution*}}{}
  \directlua{stop_proccesing_solution(flag_eat)}
}


\newcommand{\PracticeSubject}{}
\newcommand{\PracticeName}{}
\newcommand{\PracticeGroup}{}
\newcommand{\PracticeCourse}{}
\newcommand{\PracticeDate}{}


\AtBeginDocument{%
    % Метаданные:
    \title{\PracticeName}%
    \date{\PracticeDate}%
    % Настраиваем колонтитулы
    \pagestyle{fancy}%
    \lhead{\PracticeGroup, курс \PracticeCourse}%
    \chead{\PracticeSubject. Практика}%
    \rhead{\PracticeDate}%

    % Название практики
    \section*{\strtitle}
}
\mmzset{memo dir = images/cache/\jobname}


\setboolean{ShowHint}{false}
\setboolean{ShowSolution}{true}

\renewcommand{\PracticeSubject}{Дискретная математика}
\renewcommand{\PracticeName}{Эйлеровы и гамильтоновы графы}
\renewcommand{\PracticeGroup}{ПМИ}
\renewcommand{\PracticeCourse}{2}
\renewcommand{\PracticeDate}{7 февраля 2025}

\begin{document}
\begin{?}
    Верно ли, что каждый простой эйлеров граф имеет четное количество ребер? А простой эйлеров граф, построенный на четном количестве вершин? А эйлеров двудольный граф?
\end{?}
\begin{solution}{}
    Ответ на первый вопрос --- нет, граф \(K_3\) эйлеров, но имеет нечетное число ребер. Граф \(K_3 \wedge C_4\) (здесь \(\wedge\) --- склейка графов по одной вершине) также эйлеров, но также имеет нечетное число ребер. Для двудольного графа количество ребер это сумма степеней одной из долей, поэтому ребер будет четное число.
\end{solution}
\begin{?}
    Верно ли, что в эйлеровом графе для любых двух ребер $e_1$ и $e_2$, инцидентных одной и той же вершине, обязательно найдется хотя бы один эйлеров цикл, в котором эти два ребра идут одно за другим?
\end{?}
\begin{solution}{}
    Нет. Рассмотрим граф-бантик \(K_3 \wedge K_3\) и два ребра \(e_1, e_2\) примыкающее к центральной вершине бантика с одной из сторон:
    \begin{center}
        \includetikz{practice-03-solution-img-01.tikz}
    \end{center}
    Эти два ребра не будут соседними ни в одном эйлеров цикле. Однако, можно сказать следующее. Пусть \(e_1 = (v_1, u), e_2 = (u, v_2)\) и вершина \(u\) не является точкой сочленения. Удалим эти ребра из графа. Тогда, между \(v_1\) и \(v_2\) есть эйлеров путь \(P \colon v_1 \rightsquigarrow \ldots \rightsquigarrow v_2\) (граф \(G\) либо остался связным, либо содержит одну изолированную вершину \(u\)), а значит в исходном графе был эйлеров цикл \(u \rightsquigarrow P \rightsquigarrow u\).
\end{solution}
\begin{?}
    Рассмотрим связный простой регулярный граф $G$, степень любой вершины которого равна четырем. Докажите, что ребра этого графа всегда можно покрасить в два цвета (красный и синий) так, чтобы любая вершина была инцидентна ровно двум синим и ровно двум красным ребрам.
\end{?}
\begin{solution}{}
    Рассмотрим любой эйлеров цикл (замкнутый маршрут) и будем красить ребра в порядке обхода чередуя цвета. Так как при обходе мы посещаем каждую вершину ровно 2 раза, у каждой вершины будет поровну синих и красных ребер (заметим что для вершины из которой начат обход это также будет работать в силу того что ребер четное число). 
\end{solution}
\begin{?}
    Докажите, что в гамильтоновом графе точки сочленения отсутствуют.
\end{?}
\begin{solution}{}
    Используем необходимый критерий гамильтоновости: удаление точки сочленения ведет к образованию двух компонент связности, что невозможно, если в графе изначально между любыми двумя вершинами есть как минимум два пути (получающихся из гамильтонова цикла).
\end{solution}
\begin{?}
    Подсчитайте количество гамильтоновых циклов в полном графе $K_n$, построенном на $n > 2$ вершинах.
\end{?}
\begin{solution}{}
    Любая перестановка задает гамильтонов цикл. Но при этом, одному гамильтонову циклу в \(K_n\) отвечает \(2n\) перестановок порождаемых сдвигами цикла и обходом в противоположном направлении. Поэтом в \(K_n\) есть \(\frac{(n-1)!}{2}\) гамильтоновых циклов.
\end{solution}
\begin{?}
    В однокруговом турнире участвуют $2n > 4$ команд. К текущему моменту уже сыгран $n - 1$ тур. Докажите, что можно сыграть ещё как минимум два тура.
\end{?}
\begin{solution}{}
    Каждой команде предстоит еще сыграть с \(n\) командами. Таким образом, граф, где вершины это команды, а ребра соединяют те команды, которые не играли между собой, будет \(n\)-регулярным графом на \(2n\) вершинах. По теореме Дирака, такой граф гамильтонов. Возьмем любой гамильтонов цикл в этом графе и разобьём ребра на две чередующиеся группы. В силу того что число команд четно это разбиение дает нужные нам два тура.
\end{solution}
\begin{?}
    Постройте минимальный по количеству вершин и ребер граф, не имеющий точек сочленения, но не являющийся гамильтоновым. Найдите замыкание этого графа.
\end{?}
\begin{solution}{}
    См{.} \url{https://mathworld.wolfram.com/NonhamiltonianGraph.html}
\end{solution}
\begin{?}
    Докажите, что всякий самодвойственный граф \(G = \overline{G}\) содержит гамильтонов путь.
\end{?}
\begin{solution}{}
    Упорядочим вершины графа \(G\) таким образом, что
    \[
        d_{1} \leq \ldots \leq d_{n}.
    \]
    Теперь, поскольку \(G\) самодвойственный, его последовательность степеней удовлетворяет условию \(d_{i} = n - 1 - d_{n-i + 1}, i = 1, \ldots, n\). В частности это означает что верна следующая импликация:
    \[
        d_{i} \leq i - 1 < \frac{n + 1}{2} \Rightarrow d_{n + 1 - i} \geq n - i.
    \]
    Пусть \(G'\) --- граф, построенный из \(G\) добавлением одной вершины \(u\), соединённой со всеми вершинами \(G\). Тогда вершины \(G'\) удовлетворяют условию
    \[
        d_{i} \leq i < \frac{n}{2} \Rightarrow d_{n - i} \geq n - i.
    \]
    Это условие теоремы Хватала, следовательно, \(G'\) содержит гамильтонов цикл. Удаляя вершину \(u\), получаем что граф \(G\) должен содержать гамильтонов путь.
\end{solution}
\end{document}

