%
\documentclass[a4paper,12pt,twoside]{article}
\usepackage{iftex}

%% Разрешить компиляцию только с движком LuaTex
\ifLuaTeX
\else
    \newlinechar 64\relax
    \errorcontextlines -1\relax
    \immediate\write20{@
        ************************************************@
        * LuaLaTex is required to compile this document.@
        * Sorry!@
        ************************************************}%
    \batchmode\read -1 to \@tempa
\fi

%% Для русификации достаточно подключить пакет fontspec и
%% выбрать Unicode шрифт в котором есть кириллические глифы. Ниже
%% основным шрифтом выбирается Unicode версия шрифта Computer Modern с заcечками
\usepackage{fontspec}
\setmainfont{CMU Serif}
\setsansfont{CMU Sans Serif}
\setmonofont{CMU Typewriter Text}

%% В XeLaTex или LuaLaTeX альтернативой известного пакета babel является пакет polyglossia.
%% Теперь у нас будут переносы слов
\usepackage{polyglossia}
\setdefaultlanguage{russian}
\setotherlanguage{english}

\usepackage[autostyle]{csquotes} % Правильные кавычки в зависимости от языка
\usepackage{totcount}
\usepackage{setspace}

% ToDo:
% [] MathNote   : \sum[] syntax
% [] MathNote   : \tikzmatrix 
% [] Layout     : A4 geometry
% [] Layout     : Define colors
% [] References : Load knowledge
% [] References : Create simple clever refs
% [] Fix hidding enviroment

%Отключить предупреждения об кастномной использовании пакетов "You have requested package..."
\usepackage{silence}
\WarningFilter{latex}{You have requested package}

\usepackage{xparse}
\usepackage{configuration/floats}
\usepackage{configuration/layout}
\usepackage{configuration/references}
\usepackage{configuration/mathnote}

\sloppy
% Окружения для набора задач
\newcounter{boxlblcounter}  

\newcommand{\task}[1]{\fbox{\begin{minipage}{4em}\centering\it #1\end{minipage}}}
\newcommand{\makeboxlabel}[1]{\fbox{\begin{minipage}{2em}\centering\it #1\end{minipage}}\hfill}% \hfill fills the label box
\newenvironment{tasklist}
  {\begin{list}
    {\arabic{boxlblcounter}}
    {\usecounter{boxlblcounter}
     \setlength{\labelwidth}{3em}
     \setlength{\labelsep}{0em}
     \setlength{\itemsep}{2pt}
     \setlength{\leftmargin}{1.5cm}
     \setlength{\rightmargin}{2cm}
     \setlength{\itemindent}{0em} 
     \let\makelabel=\makeboxlabel
    }
  }{\end{list}}


\newboolean{ShowHint}
\newboolean{ShowSolution}

% Подсказка
\newcommand{\hint}[1]{\ifthenelse{\boolean{ShowHint}}{\noindent\rotatebox[origin=c]{180}{\noindent
\begin{minipage}[t]{\linewidth} \noindent \it Указание: #1 \end{minipage}}}{}}

% solution: окружение для набора решений
% solution*: форсировано показывает решение, независимо от флага
\makeatletter
\newenvironment{solution*}[1][\text{Решение}]{ 
  \par
  \pushQED{\qed}%
  \normalfont
  \topsep0pt \partopsep3pt
  \trivlist
  \item[\hskip\labelsep
    \itshape
    #1\@addpunct{.}]\ignorespaces
}{
  \ifthenelse{\boolean{ShowSolution}}{
    \popQED\endtrivlist\@endpefalse
    \addvspace{6pt plus 6pt} % some space after
  }{\end{hidden}}
}
\makeatother

%https://tex.stackexchange.com/questions/533218/hiding-an-environment-that-contains-minted-code
%https://tex.stackexchange.com/questions/38150/in-lualatex-how-do-i-pass-the-content-of-an-environment-to-lua-verbatim
\RequirePackage{luacode}
\begin{luacode*}
do 
    function eat_buffer(buf)
        i,j = string.find(buf,"\\end{solution}")
        if i==nil then return "" else return nil end
    end
    function start_proccesing_solution(eat)
        if eat then luatexbase.add_to_callback('process_input_buffer', eat_buffer, 'eat_buffer') end
    end
    function stop_proccesing_solution(eat)
        if eat then luatexbase.remove_from_callback('process_input_buffer', 'eat_buffer') end
    end
end
\end{luacode*}
%https://tex.stackexchange.com/questions/537219/conditionals-inside-newcommand-with-empty-argument
%https://tex.stackexchange.com/questions/63223/xparse-empty-arguments
\ExplSyntaxOn
\DeclareExpandableDocumentCommand{\IfNoValueOrEmptyTF}{mmm}{\IfNoValueTF{#1}{#2}{\tl_if_empty:nTF {#1} {#2} {#3}}}
\NewDocumentCommand{\DefaultName}{mm}{\IfNoValueOrEmptyTF{#1}{#2}{#1}}
\ExplSyntaxOff

% WARNING!
% Buggy: one have always have
% \begin{solution}{NAME}
%   ....
% \end{solution}
% EVEN if NAME is empty
% if one starts some text on the same line as \begin{solution} or \end{solution} it could'not be hidden
\newenvironment{solution}[1]%
{
  \ifthenelse{\boolean{ShowSolution}}
    {\begin{solution*}[\DefaultName{#1}{\textit{Решение}}]\directlua{flag_eat=false}}
    {\directlua{flag_eat=true}}
  \directlua{start_proccesing_solution(flag_eat)}
}
{
  \ifthenelse{\boolean{ShowSolution}}{\end{solution*}}{}
  \directlua{stop_proccesing_solution(flag_eat)}
}


\newcommand{\PracticeSubject}{}
\newcommand{\PracticeName}{}
\newcommand{\PracticeGroup}{}
\newcommand{\PracticeCourse}{}
\newcommand{\PracticeDate}{}


\AtBeginDocument{%
    % Метаданные:
    \title{\PracticeName}%
    \date{\PracticeDate}%
    % Настраиваем колонтитулы
    \pagestyle{fancy}%
    \lhead{\PracticeGroup, курс \PracticeCourse}%
    \chead{\PracticeSubject. Практика}%
    \rhead{\PracticeDate}%

    % Название практики
    \section*{\strtitle}
}
\mmzset{memo dir = images/cache/\jobname}


\setboolean{ShowHint}{true}
\setboolean{ShowSolution}{true}

\renewcommand{\PracticeSubject}{Дискретная математика}
\renewcommand{\PracticeName}{Теорма Кантора-Бернштейна}
\renewcommand{\PracticeGroup}{ПАДИИ}
\renewcommand{\PracticeCourse}{1}
\renewcommand{\PracticeDate}{6 февраля 2025}


\begin{document}
\begin{?}
    Докажите, что если отрезок разбит на две части, то хотя бы одна из них равномощна отрезку.
\end{?}
\begin{solution}{}
    Рассмотрим отрезок и его разбиение на два множества \(A\) и \(B\). Сделаем биекцию отрезка с квадратом \([0; 1] \times [0; 1]\), разбиение отрезка дает разбиение квадрата на два множества \(X\) и \(Y\), причем \(|A| = |X|\) и \(|B| = |Y|\). С другой стороны если \(X\) (соответственно \(Y\)) содержит какой-то отрезок, то \(X\) (соотв. \(Y\)), а значит и \(A\) (соотв. \(B\)) равномощны отрезку по т. Кантора-Бернштейна. С другой стороны, если ни \(X\), ни \(Y\) не содержат отрезка, то на каждой прямой вида \(x = a, a \in [0; 1]\) есть точка из множества \(X\), а значит \(X\) содержит в себе подмножество равномощное \([0; 1]\), противоречие.
\end{solution}
\begin{?}
    Пусть \(f\) --- взаимно-однозначное соответствие между \(A\) и некоторым подмножеством множества \(B\), а \(g\) --- взаимно-однозначное соответствие между \(B\) и некоторым подмножеством множества \(A\). Докажите, что можно так разбить множество \(A\) на непересекающиеся части \(A'\) и \(A''\), а множество \(B\) --- на непересекающиеся части \(B'\) и \(B''\), что \(f\) осуществляет взаимно-однозначное соответствие между \(A'\) и \(B'\), а \(g\) --- между \(A''\) и \(B''\).
\end{?}
\begin{solution}{}
    Построим множества \(A'\) и \(B'\) следующим образом. Положим \(A_0 = A\), \(B_0 = B\) и 
    \begin{align*}
        & B_1 = f(A_0), \quad A_1 = g(B_0) \\
        & B_2 = f(A_1), \quad A_2 = g(B_1) \\
        & \ldots
    \end{align*}
    Легко показать по индукции, что \(A_0 \supset A_1 \supset \ldots \) и \(B_0 \supset B_1 \supset \ldots \). Теперь положим:
    \[
        \hat{A}_n = A_{n} \setminus A_{n+1} \text{ и } \hat{B}_n = B_n \setminus B_{n+1} \quad n \in \N_0 
    \]
    Тогда, если считать что \(\mathcal{A}_{\infty} = \bigcap _{n = 0}^{+\infty}  \hat{A}_n\) и \(\mathcal{B}_{\infty} = \bigcap _{n = 0}^{+\infty}  \hat{B}_n\), то
    \begin{align*}
        &A = \bigcup_{n = 0}^{+\infty}  \hat{A}_n \cup \mathcal{A}_{\infty} \\
        &B = \bigcup_{n = 0}^{+\infty}  \hat{B}_n \cup \mathcal{B}_{\infty} 
    \end{align*}
    Более того \(f(\mathcal{A}_{\infty}) =  \mathcal{B}_{\infty}\) и \(g(\mathcal{A}_{\infty}) =  \mathcal{A}_{\infty}\). Действительно, \(\mathcal{B}_{\infty} \subset B_1\) и значит лежит в образе. Теперь пусть есть какая-то точка \(a\in \mathcal{A}_{\infty}\), что \(f(a) \not\in \mathcal{B}_{\infty}\). Тогда \(f(a) \not\in B_{k}\) начиная с некоторого \(k \in \mathbb{N}\), но тогда \(a \not\in A_{k - 1}\), противоречие. 

    Нетрудно видеть что \(f\) осуществляет биекцию между 
    \[
        A'_{f} = \bigcup_{n = 0}^{\infty} \hat{A}_{2n} \text{ и } B'_{f} = \bigcup_{n = 0}^{\infty} \hat{B}_{2n+1}
    \]
    в то время как \(g\) осуществляет биекцию между 
    \[
        B'_{g} = \bigcup_{n = 0}^{\infty} \hat{B}_{2n} \text{ и } A'_{g} = \bigcup_{n = 0}^{\infty} \hat{A}_{2n+1}
    \]
    добавляя \(\mathcal{A}_{\infty}\) и \(\mathcal{B}_{\infty}\) к паре \((A_f', B_f'\) или к паре \(A_g', B_g'\) получаем нужное разбиение.
\end{solution}

\noindent \textbf{Определение}. Множество \(E \subset \R\) называется замкнутым, если оно содержит все свои предельные точки, т.е. предел любой сходящейся последовательности \(x_1, \ldots, x_i, \ldots\) элементов \(E\) также лежит в \(E\).

\noindent \textbf{Определение}. Точка \(a \in E \subset \R\) называется изолированной, если существует \(\varepsilon > 0\), такое что \(B_{\varepsilon}(a) \cap E = \set{a}\)\footnote{\(B_{\varepsilon}(a) = \set{x \in \R | \abs{x - a} < \varepsilon}\)}.

\begin{?}
    Покажите, что любое непустое замкнутое множество \(A \subset \mathbb R\) без изолированных точек имеет мощность континуума.
\end{?}
\begin{solution}{}
    Так как \(A\) не содержит изолированных точек, в любой окрестности любой точки \(A\) содержится бесконечно много точек из \(A\). Пусть \(x_0 \in A\). Тогда в \(B_{\delta_0}(x_0), \delta_0 = 1\) выберем какие-нибудь две точки \(x_1^1, x_1^2\), и возьмем \(\delta_1 < \lfrac{1}{2}\), так чтобы \(B_{\delta}(x_1^1) \cap B_{\delta}(x_1^2) = \varnothing\). Продолжим итеративно: для точки \(x_i^j, i \in \mathbb \N, j = 1, \ldots, 2^i\) возьмем в окрестности радиуса \(\delta_i < \min\set{\lfrac{1}{2^i}, \delta_{i - 1}}\) две точки. В итоге получим следующую картину.
    \begin{center}
        \includetikz{practice-03-solution-img-01.tikz}
    \end{center}
    Теперь, с каждой бесконечной бинарной последовательностью свяжем последовательность точек следующим образом. Пусть \(b = (b_1, b_2, \ldots, b_i, \ldots)\) это бесконечная бинарная последовательность, \(b^{i} = (b_1, b_2, \ldots, b_i)\), её срезка, \(N(b^i)\) --- число, получаемое переводом \(b^{i}\) в десятичную запись (причем запись идёт справа налево, т.е. \(001_2 \mapsto 1\)). Тогда по последовательности \(b\) строим последовательность \(\set{x_i^{N(b^i)}}_{i \in \mathbb \N}\) --- это можно представить себе как спуск по бесконечному бинарному дереву. Причем каждая такая последовательность будет сходящейся (так как окрестности вкладываются в друг друга, а их радиус стремится к 0) к своему уникальному пределу (так как окрестности отделены), и их будет континуум штук. Откуда \(A\) имеет как минимум континуум различных точек. Так как \(A \subset \R\), то по теореме Кантора-Бернштейна \(A\) имеет мощность континуума.
\end{solution}
\begin{?}
    Покажите, что любое замкнутое множество \(A \subseteq \mathbb R\) либо конечно, либо счётно, либо имеет мощность континуума.
\end{?}
\begin{solution}{}
    Следствие предыдущей задачи. Если из замкнутого множества удалить изолированные точки, оно останется замкнутым. При этом, у любого множества есть не более чем счетное число изолированных точек. 
\end{solution}
\begin{?}
    Докажите с помощью теории множеств, что \(n < 2^n\) для всех натуральных \(n\).
\end{?}
\begin{solution}{}
    Следствие теоремы Кантора о том что между \(A\) и \(2^A\) нет биекции. При этом очевидно что \(A\) вкладывается инъективно в \(2^A\)
\end{solution}

\end{document}

