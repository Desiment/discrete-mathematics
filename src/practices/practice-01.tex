%
\documentclass[a4paper,12pt,twoside]{article}
\usepackage{iftex}

%% Разрешить компиляцию только с движком LuaTex
\ifLuaTeX
\else
    \newlinechar 64\relax
    \errorcontextlines -1\relax
    \immediate\write20{@
        ************************************************@
        * LuaLaTex is required to compile this document.@
        * Sorry!@
        ************************************************}%
    \batchmode\read -1 to \@tempa
\fi

%% Для русификации достаточно подключить пакет fontspec и
%% выбрать Unicode шрифт в котором есть кириллические глифы. Ниже
%% основным шрифтом выбирается Unicode версия шрифта Computer Modern с заcечками
\usepackage{fontspec}
\setmainfont{CMU Serif}
\setsansfont{CMU Sans Serif}
\setmonofont{CMU Typewriter Text}

%% В XeLaTex или LuaLaTeX альтернативой известного пакета babel является пакет polyglossia.
%% Теперь у нас будут переносы слов
\usepackage{polyglossia}
\setdefaultlanguage{russian}
\setotherlanguage{english}

\usepackage[autostyle]{csquotes} % Правильные кавычки в зависимости от языка
\usepackage{totcount}
\usepackage{setspace}

% ToDo:
% [] MathNote   : \sum[] syntax
% [] MathNote   : \tikzmatrix 
% [] Layout     : A4 geometry
% [] Layout     : Define colors
% [] References : Load knowledge
% [] References : Create simple clever refs
% [] Fix hidding enviroment

%Отключить предупреждения об кастномной использовании пакетов "You have requested package..."
\usepackage{silence}
\WarningFilter{latex}{You have requested package}

\usepackage{xparse}
\usepackage{configuration/floats}
\usepackage{configuration/layout}
\usepackage{configuration/references}
\usepackage{configuration/mathnote}

\sloppy
% Окружения для набора задач
\newcounter{boxlblcounter}  

\newcommand{\task}[1]{\fbox{\begin{minipage}{4em}\centering\it #1\end{minipage}}}
\newcommand{\makeboxlabel}[1]{\fbox{\begin{minipage}{2em}\centering\it #1\end{minipage}}\hfill}% \hfill fills the label box
\newenvironment{tasklist}
  {\begin{list}
    {\arabic{boxlblcounter}}
    {\usecounter{boxlblcounter}
     \setlength{\labelwidth}{3em}
     \setlength{\labelsep}{0em}
     \setlength{\itemsep}{2pt}
     \setlength{\leftmargin}{1.5cm}
     \setlength{\rightmargin}{2cm}
     \setlength{\itemindent}{0em} 
     \let\makelabel=\makeboxlabel
    }
  }{\end{list}}


\newboolean{ShowHint}
\newboolean{ShowSolution}

% Подсказка
\newcommand{\hint}[1]{\ifthenelse{\boolean{ShowHint}}{\noindent\rotatebox[origin=c]{180}{\noindent
\begin{minipage}[t]{\linewidth} \noindent \it Указание: #1 \end{minipage}}}{}}

% solution: окружение для набора решений
% solution*: форсировано показывает решение, независимо от флага
\makeatletter
\newenvironment{solution*}[1][\text{Решение}]{ 
  \par
  \pushQED{\qed}%
  \normalfont
  \topsep0pt \partopsep3pt
  \trivlist
  \item[\hskip\labelsep
    \itshape
    #1\@addpunct{.}]\ignorespaces
}{
  \ifthenelse{\boolean{ShowSolution}}{
    \popQED\endtrivlist\@endpefalse
    \addvspace{6pt plus 6pt} % some space after
  }{\end{hidden}}
}
\makeatother

%https://tex.stackexchange.com/questions/533218/hiding-an-environment-that-contains-minted-code
%https://tex.stackexchange.com/questions/38150/in-lualatex-how-do-i-pass-the-content-of-an-environment-to-lua-verbatim
\RequirePackage{luacode}
\begin{luacode*}
do 
    function eat_buffer(buf)
        i,j = string.find(buf,"\\end{solution}")
        if i==nil then return "" else return nil end
    end
    function start_proccesing_solution(eat)
        if eat then luatexbase.add_to_callback('process_input_buffer', eat_buffer, 'eat_buffer') end
    end
    function stop_proccesing_solution(eat)
        if eat then luatexbase.remove_from_callback('process_input_buffer', 'eat_buffer') end
    end
end
\end{luacode*}
%https://tex.stackexchange.com/questions/537219/conditionals-inside-newcommand-with-empty-argument
%https://tex.stackexchange.com/questions/63223/xparse-empty-arguments
\ExplSyntaxOn
\DeclareExpandableDocumentCommand{\IfNoValueOrEmptyTF}{mmm}{\IfNoValueTF{#1}{#2}{\tl_if_empty:nTF {#1} {#2} {#3}}}
\NewDocumentCommand{\DefaultName}{mm}{\IfNoValueOrEmptyTF{#1}{#2}{#1}}
\ExplSyntaxOff

% WARNING!
% Buggy: one have always have
% \begin{solution}{NAME}
%   ....
% \end{solution}
% EVEN if NAME is empty
% if one starts some text on the same line as \begin{solution} or \end{solution} it could'not be hidden
\newenvironment{solution}[1]%
{
  \ifthenelse{\boolean{ShowSolution}}
    {\begin{solution*}[\DefaultName{#1}{\textit{Решение}}]\directlua{flag_eat=false}}
    {\directlua{flag_eat=true}}
  \directlua{start_proccesing_solution(flag_eat)}
}
{
  \ifthenelse{\boolean{ShowSolution}}{\end{solution*}}{}
  \directlua{stop_proccesing_solution(flag_eat)}
}


\newcommand{\PracticeSubject}{}
\newcommand{\PracticeName}{}
\newcommand{\PracticeGroup}{}
\newcommand{\PracticeCourse}{}
\newcommand{\PracticeDate}{}


\AtBeginDocument{%
    % Метаданные:
    \title{\PracticeName}%
    \date{\PracticeDate}%
    % Настраиваем колонтитулы
    \pagestyle{fancy}%
    \lhead{\PracticeGroup, курс \PracticeCourse}%
    \chead{\PracticeSubject. Практика}%
    \rhead{\PracticeDate}%

    % Название практики
    \section*{\strtitle}
}
\mmzset{memo dir = images/cache/\jobname}


\setboolean{ShowHint}{true}
\setboolean{ShowSolution}{false}

\renewcommand{\PracticeSubject}{Дискретная математика}
\renewcommand{\PracticeName}{Графы. Начало}
\renewcommand{\PracticeGroup}{ПМИ}
\renewcommand{\PracticeCourse}{2}
\renewcommand{\PracticeDate}{20 янвая 2025}


\begin{document}
\subsection*{Вспоминаем что было}
\begin{itemize}[noitemsep, parsep=0pt,topsep=0pt]
        \item Граф это пара $G = (V, E)$, где $V$ --- множество вершин, а $E$ --- множество ребер.
        \item Степенью вершины $\mathrm{deg}\,v$ называется число ребер $e$ инцидентных этой вершине (т.е. ребер вида $e = \{v, \cdot\})$
        \item Минимальная среди всех вершин степень обозначается $\delta(G)$, а максимальная --- $\Delta(G)$.
\end{itemize}

\noindent \textbf{Лемма о рукопожатиях}. В любом графе верно тождество:
$$
    \sum_{v \in V} \deg v = 2 \cdot |E|
$$

\begin{itemize}[noitemsep, parsep=0pt,topsep=0pt]
    \item Маршрутом называется чередующаяся последовательность вершин и ребер $v_1 e_1 v_2 e_2 \ldots e_{\ell} v_{\ell+1}$, где $e_i = \{v_i, v_{i+1}\}$. Число $\ell$ называется длинной маршрута.
    \item Маршрут называется замкнутым если $v_1 = v_n$.
    \item Путем называется любой маршрут в котором все  ребра различны. Простым путем называется путь в котором различных все вершины.
    \item Циклом в графе называется циклическая чередующаяся последовательность вершин и ребер $v_1 e_1 \ldots v_{\ell} e_{\ell}$, причем $e_\ell = \{v_\ell, v_1\}$. Цикл называется простым если все вершины $v_1, \ldots, v_\ell$ различны.  
\end{itemize}

Граф называется \textbf{связным}, если между любой парой вершин существует маршрут. Связный ациклический граф называется \textbf{деревом}.

\noindent \textbf{Лемма.} В любом дереве на $n$ вершинах ровно $n-1$ ребро.

\subsection*{Задачи}
\begin{?}
    Пусть $G$ простой граф, построенный на 9 вершинах. Предположим, что сумма степеней вершин графа $G$ больше или равна 27. Правда ли, что в таком графе обязательно существует вершина, степень которой больше или равна 4?
\end{?}
\begin{?}
    В графе на $n$ вершинах все степени вершин не меньше $k$. При каком наименьшем $k$ отсюда следует, что граф связен?
\end{?}
\begin{?}    
    Докажите, что в любом простом графе, построенном на $n > 2$ вершинах, существуют по крайней мере две вершины с одинаковыми степенями. Остается ли верным это утверждение для мультиграфа? Для графа без петель?
\end{?}
\begin{?}
    В связном графе нашлись два простых пути максимальной длины. Докажите, что у них есть хотя бы одна общая вершина.
\end{?}
\begin{?}
    В выпуклом $n$-угольнике провели несколько диагоналей, не имеющих общих внутренних точек, которые
    разбили его на треугольники (это называется триангулировать многоугольник).
    \begin{itemize}[noitemsep, parsep=0pt,topsep=0pt]
        \item Докажите, что получилось ровно $n-2$ треугольника.
        \item Пусть каждый треугольник --- вершина графа, а две вершины смежны, если треугольники имеют общую
        сторону. Докажите, что получилось дерево.
    \end{itemize}
\end{?}
\begin{?}
    Все города страны (в том числе столица) соединены кольцевой железной дорогой. Кроме того, столица соединена отдельными линиями с каждым из городов, кроме соседей по кольцу. Правительство решило разделить железнодорожную сеть между двумя компаниями так, чтобы, пользуясь дорогами любой из компаний, можно было доехать от любого города до любого другого. Можно ли выполнить это решение?
\end{?}
\begin{?}
    Вершины дерева $T$ покрашены в красный и синий цвет так, что любые две смежные вершины разноцветны, и красных не меньше, чем синих. Докажите, что у $T$ есть красный лист.
\end{?}
\end{document}
