%
\documentclass[a4paper,12pt,twoside]{article}
\usepackage{iftex}

%% Разрешить компиляцию только с движком LuaTex
\ifLuaTeX
\else
    \newlinechar 64\relax
    \errorcontextlines -1\relax
    \immediate\write20{@
        ************************************************@
        * LuaLaTex is required to compile this document.@
        * Sorry!@
        ************************************************}%
    \batchmode\read -1 to \@tempa
\fi

%% Для русификации достаточно подключить пакет fontspec и
%% выбрать Unicode шрифт в котором есть кириллические глифы. Ниже
%% основным шрифтом выбирается Unicode версия шрифта Computer Modern с заcечками
\usepackage{fontspec}
\setmainfont{CMU Serif}
\setsansfont{CMU Sans Serif}
\setmonofont{CMU Typewriter Text}

%% В XeLaTex или LuaLaTeX альтернативой известного пакета babel является пакет polyglossia.
%% Теперь у нас будут переносы слов
\usepackage{polyglossia}
\setdefaultlanguage{russian}
\setotherlanguage{english}

\usepackage[autostyle]{csquotes} % Правильные кавычки в зависимости от языка
\usepackage{totcount}
\usepackage{setspace}

% ToDo:
% [] MathNote   : \sum[] syntax
% [] MathNote   : \tikzmatrix 
% [] Layout     : A4 geometry
% [] Layout     : Define colors
% [] References : Load knowledge
% [] References : Create simple clever refs
% [] Fix hidding enviroment

%Отключить предупреждения об кастномной использовании пакетов "You have requested package..."
\usepackage{silence}
\WarningFilter{latex}{You have requested package}

\usepackage{xparse}
\usepackage{configuration/floats}
\usepackage{configuration/layout}
\usepackage{configuration/references}
\usepackage{configuration/mathnote}

\sloppy
% Окружения для набора задач
\newcounter{boxlblcounter}  

\newcommand{\task}[1]{\fbox{\begin{minipage}{4em}\centering\it #1\end{minipage}}}
\newcommand{\makeboxlabel}[1]{\fbox{\begin{minipage}{2em}\centering\it #1\end{minipage}}\hfill}% \hfill fills the label box
\newenvironment{tasklist}
  {\begin{list}
    {\arabic{boxlblcounter}}
    {\usecounter{boxlblcounter}
     \setlength{\labelwidth}{3em}
     \setlength{\labelsep}{0em}
     \setlength{\itemsep}{2pt}
     \setlength{\leftmargin}{1.5cm}
     \setlength{\rightmargin}{2cm}
     \setlength{\itemindent}{0em} 
     \let\makelabel=\makeboxlabel
    }
  }{\end{list}}


\newboolean{ShowHint}
\newboolean{ShowSolution}

% Подсказка
\newcommand{\hint}[1]{\ifthenelse{\boolean{ShowHint}}{\noindent\rotatebox[origin=c]{180}{\noindent
\begin{minipage}[t]{\linewidth} \noindent \it Указание: #1 \end{minipage}}}{}}

% solution: окружение для набора решений
% solution*: форсировано показывает решение, независимо от флага
\makeatletter
\newenvironment{solution*}[1][\text{Решение}]{ 
  \par
  \pushQED{\qed}%
  \normalfont
  \topsep0pt \partopsep3pt
  \trivlist
  \item[\hskip\labelsep
    \itshape
    #1\@addpunct{.}]\ignorespaces
}{
  \ifthenelse{\boolean{ShowSolution}}{
    \popQED\endtrivlist\@endpefalse
    \addvspace{6pt plus 6pt} % some space after
  }{\end{hidden}}
}
\makeatother

%https://tex.stackexchange.com/questions/533218/hiding-an-environment-that-contains-minted-code
%https://tex.stackexchange.com/questions/38150/in-lualatex-how-do-i-pass-the-content-of-an-environment-to-lua-verbatim
\RequirePackage{luacode}
\begin{luacode*}
do 
    function eat_buffer(buf)
        i,j = string.find(buf,"\\end{solution}")
        if i==nil then return "" else return nil end
    end
    function start_proccesing_solution(eat)
        if eat then luatexbase.add_to_callback('process_input_buffer', eat_buffer, 'eat_buffer') end
    end
    function stop_proccesing_solution(eat)
        if eat then luatexbase.remove_from_callback('process_input_buffer', 'eat_buffer') end
    end
end
\end{luacode*}
%https://tex.stackexchange.com/questions/537219/conditionals-inside-newcommand-with-empty-argument
%https://tex.stackexchange.com/questions/63223/xparse-empty-arguments
\ExplSyntaxOn
\DeclareExpandableDocumentCommand{\IfNoValueOrEmptyTF}{mmm}{\IfNoValueTF{#1}{#2}{\tl_if_empty:nTF {#1} {#2} {#3}}}
\NewDocumentCommand{\DefaultName}{mm}{\IfNoValueOrEmptyTF{#1}{#2}{#1}}
\ExplSyntaxOff

% WARNING!
% Buggy: one have always have
% \begin{solution}{NAME}
%   ....
% \end{solution}
% EVEN if NAME is empty
% if one starts some text on the same line as \begin{solution} or \end{solution} it could'not be hidden
\newenvironment{solution}[1]%
{
  \ifthenelse{\boolean{ShowSolution}}
    {\begin{solution*}[\DefaultName{#1}{\textit{Решение}}]\directlua{flag_eat=false}}
    {\directlua{flag_eat=true}}
  \directlua{start_proccesing_solution(flag_eat)}
}
{
  \ifthenelse{\boolean{ShowSolution}}{\end{solution*}}{}
  \directlua{stop_proccesing_solution(flag_eat)}
}


\newcommand{\PracticeSubject}{}
\newcommand{\PracticeGroup}{}
\newcommand{\PracticeCourse}{}
\newcommand{\HWname}{}
\newcommand{\HWnumber}{}
\newcommand{\Deadline}{}

\AtBeginDocument{%
    % Метаданные:
    \title{\HWname}%
    \date{\today}%
    % Настраиваем колонтитулы
    \pagestyle{fancy}%
    \lhead{\PracticeGroup, курс \PracticeCourse}%
    \chead{\PracticeSubject. ДЗ №\HWnumber}%
    \rhead{Дедлайн: \Deadline}%
    % Название практики
    \section*{\strtitle}
}
\mmzset{memo dir = images/cache/\jobname}


\setboolean{ShowHint}{false}
\setboolean{ShowSolution}{false}

\renewcommand{\PracticeSubject}{Дискретная математика}
\renewcommand{\PracticeGroup}{ПМИ}
\renewcommand{\PracticeCourse}{2}
\renewcommand{\HWname}{ДЗ №3, эйлеровы и гамильтоновы циклы}
\renewcommand{\HWnumber}{3}
\renewcommand{\Deadline}{14 февраля в 23:59}


\begin{document}
\begin{?}[Разминочная]\ \\
    \task{0.5 балла} Верно ли, что если в простом графе \( G \) существует эйлеров цикл, то в нём существует и гамильтонов цикл? Верно ли обратное утверждение, а именно, что если в простом графе существует гамильтонов цикл, то в нём существует и эйлеров цикл?
\end{?}
\begin{?}\ \\
    \task{1 балл} Докажите, что в эйлеровом графе мосты отсутствуют.
\end{?}
\begin{?}\ \\
    \task{1 балл}  Докажите, что в графе, изображённом на рисунке ниже, гамильтонов цикл отсутствует.    
    \begin{center}
        \begin{tikzpicture}[node distance=2cm, every node/.style={circle, fill=black, inner sep=1.5pt}]
            % Define nodes
            \node (A) at (0, 0) {};
            \node (B) at (2, 2) {};
            \node (C) at (4, 0) {};
            \node (D) at (2, -2) {};
            \node (E) at (1, 1) {};
            \node (F) at (3, 1) {};
            \node (G) at (3, -1) {};
            \node (H) at (1, -1) {};
            \node (I) at (2, 0) {};
            \node (Y) at (6, 0) {};
            \node (W) at (-2, 0) {};
        
            % Draw edges
            \draw (Y) -- (B) -- (W) -- (D) -- (Y);
            \draw (A) -- (B) -- (C) -- (D) -- (A);
            \draw (A) -- (E);
            \draw (Y) -- (C);
            \draw (W) -- (A);
            \draw (B) -- (F);
            \draw (C) -- (G);
            \draw (D) -- (H);
            \draw (E) -- (I);
            \draw (F) -- (I);
            \draw (G) -- (I);
            \draw (H) -- (I);
        \end{tikzpicture}        
    \end{center}
\end{?}
\begin{?}\ \\
    \task{2 балла} Имеется кусок проволоки длиной 12 сантиметров. На какое минимальное количество кусков его следует разрезать, чтобы из этих кусков можно было бы изготовить каркас кубика размерами \( 1 \times 1 \times 1 \) при условии, что проволоку в процессе изготовления кубиков можно сгибать?
\end{?}
\begin{?}\ \\
    \task{1.5 балла} На поединок собрались \( 2n \) дуэлянтов. Некоторые пары дуэлянтов ненавидят друг друга, и это чувство всегда взаимно. Оказалось, что если какие-то два дуэлянта не ненавидят друг друга, то каждый из остальных ненавидит хотя бы одного из них, а кто-то ещё и обязательно ненавидит их обоих. Докажите, что всех дуэлянтов можно разбить на пары для дуэли так, чтобы каждая пара противников ненавидела друг друга.
\end{?}
\begin{?}
    Пусть \(d_1 \leq \ldots \leq d_n\) последовательность степеней вершин графа \(G\). 
    \begin{tasklist}
        \item[1] Докажите что если \(d_i + d_{n - i} \geq n\) для всех \(i = 1, \ldots, n - 1\), то \(G\) --- гамильтонов.
        \item[1] Докажите что если \(d_i + d_{n + 1 - i} \geq n - 1\) для всех \(i = 1, \ldots, n\), то в \(G\) есть гамильтонов путь.
    \end{tasklist}
\end{?}
\begin{?}\ \\
    \task{1 балл} Постройте граф, в котором  \(\delta(G) = \frac{n - 1}{2}\), но он не является гамильтоновым.
\end{?}
\begin{?}\ \\
    \task{1 балл} Посчитайте \(\mathrm{ec}(G)\) для какой-нибудь сбалансированной ориентации графа \(K_5\).
\end{?}
\end{document}